%\documentclass[pra,showpacs,twocolumn]{revtex4}


\documentclass[preprint,12pt]{article}
%%%%%%%%%%%%%%%%%%%%%%%%%%%%%%%%%%%%%%%%%%%%%%%%%%%%%%%%%%%%%%%%%%%%%%%%
\usepackage{amssymb}
\usepackage{amsmath}
\usepackage{graphicx}
\usepackage{multirow}

\usepackage{authblk}

\setcounter{MaxMatrixCols}{10}
%TCIDATA{OutputFilter=LATEX.DLL}
%TCIDATA{Version=5.50.0.2953}
%TCIDATA{<META NAME="SaveForMode" CONTENT="1">}
%TCIDATA{BibliographyScheme=Manual}
%TCIDATA{LastRevised=Thursday, March 07, 2019 14:11:39}
%TCIDATA{<META NAME="GraphicsSave" CONTENT="32">}
%TCIDATA{Language=American English}

%\input{tcilatex}
\begin{document}

\title{Ionization of molecules by multicharged bared ions by using the
stoichiometric model}
\author{author}
\affil{Instituto de Astronom\'{\i}a y F\'{\i}sica del Espacio (CONICET-UBA).
Casilla de correo 67, sucursal 28 (C1428EGA) Buenos Aires, Argentina.}
%\author{author}
%\affil{Instituto de Astronom\'{\i}a y F\'{\i}sica del Espacio (CONICET-UBA).
%Casilla de correo 67, sucursal 28 (C1428EGA) Buenos Aires, Argentina.}
%\affil{Fac. de Ciencias Exactas y Naturales, Universidad de Buenos Aires}
\date{\today }



\maketitle

\begin{abstract}
\end{abstract}

%\pacs{79.20.Rf, 68.49.Bc, 34.20.Cf,34.50.-s,34.35.+a}

\section{Introduction}

The damage caused by the impact of multicharged heavy projectile has 
become a field of interest because of its implementation in ion-beam 
cancer therapy. The effectiveness of the radiation depends on the 
choice of the ions. In particular, theoretical and experimental studies 
with different projectiles have concluded that charged carbon ions 
could be the most suitable ions to use. The study of such systems 
represent a challenge from the theoretical point of view. The most 
widely used method to compute ionization of multicharged atoms is the 
first Born approximation. 
This perturbative method warrants the $Z^{2}$ laws, where $Z$ is the 
projectile charge. However, the damage concentrated in the vicinities 
of the Bragg peak correspond to energies of hundreds of KeV/amu. 
Precisely in this region, often refered to as the intermediate energy 
region, the Born approximation starts to fail. Another theoretical 
issue arises from the targets themselves; we are dealing with complex 
molecules. This article deals precisely with these two aspects. 
First, we perform more appropriate calculations on the primary damage 
mechanism, i.e., atomic ionization by multicharged ions, which can 
replace the Born results. Second, we inspect and use an stoichiometric 
model, which reduces a molecule to a sum of atomic processes quantities 
weighted by the numbers of such atom in the molecule.

To overcome the first perturbative approximation limitations and 
since the projectiles are multicharged ions, we resort to the 
Continuum Distorted Wave-Eikonal Initial State (CDW), which includes 
higher perturbative corrections. We start from the premise that the 
ionization process is the mechanism that deposits the most significant 
amount of primary energy. This process produces an electron-energy 
spectrum that needs to be integrated. Moreover, the ejected electrons 
become a new source of local damage. These secondary electrons are 
included in Monte Carlo simulations and hence their behavior must be
investigated. With this end in mind, we calculate not only the 
ionization cross-sections but also energy and mean angular distributions 
of the emitted electrons.

The molecular structure complexity of the target is dealt with by 
implementing the simplest stoichiometric model (SSM); we assume that 
the molecule is composed of isolated independent atoms. In all cases, 
the total cross-section is simply the sum of the cross-section of each 
atom.

Then, by implementing the SSM and the CDW --instead of the first Born 
approximation--, we calculate ionization cross-section of several 
molecules (see Table~\ref{tab:families}) by the impact of antiprotons,
H$^{+}$, He$^{2+}$, Be$^{4+}$, C$^{6+}$,\ and O$^{8+}$. Furthermore, 
in Section 3, we investigate different DNA and RNA molecules
such as adenine, timine, cytosine, guanine, uracil, and also DNA 
backbone.

The results are processed to test the Toburen scaling rule, which 
states that the ratio between the ionization cross-section and the 
number of valence electrons (outer electrons) in terms of the 
projectile velocity can be arranged in a narrow universal band. 
We have also applied this rule to several hydrocarbons and nucleobases. 
In Section 3.2, we will prove that the width of the resulting universal 
band can be significantly reduced if we redefine the effective number 
of valence electrons.

It is well known that the residual electrons from the ionization 
process cause significant biological damage. To inspect this mechanism, 
in Section 3.3 and 3.4, we calculated the mean electron energy an 
angular distributions. Surprisingly, we found a substantial dependence 
of the charged projectile, which is unexpected in the first Born 
approximation. 

The hefty charge projectiles are dealt with the CDW. However, the 
stoichiometric model used seems to be very simplistic. The approximation
considers each atom as neutral, which is not correct. In Section 3.5, 
we used the molecular electronic structure code 
{\sc gamess}~\cite{gamess} to calculate the excess or defect of 
electron density one each atom. Then, the simple stoichiometric model is
modified to account for the departure from neutrality of the atoms. 
We find that for the DNA molecules this modification does not introduce 
a substantial change.

\section{Theory: Ionization of Atoms}

In our study, we will consider six atoms $\alpha=$ H, C, N, O, P, and S. 
Most of the organic molecules are composed of these atoms. Some 
particular molecules also include halogen atoms such as fluor and 
bromine; ionization cross-sections of these elements have been 
previously published~\cite{miraglia2008,miraglia2009}.

The total ionization cross-sections of these atoms were calculated using 
the CDW. The initial bound and final continuum radial wave functions 
were obtained by using the {\sc radialf} code, developed by Salvat and 
co-workers, and a Hartree-Fock potential obtained from the Depurated 
Inversion Model~\cite{mendez2016,mendez2018}. 
We used a few thousand pivot points to solve the Schr\"{o}dinger 
equation, depending on the number of oscillations of the continuum 
state. The radial integration was performed using the 
cubic spline technique. The number of angular momenta considered 
varied from 8, at very low ejected-electron energies, up to 30, 
for the highest energies considered. The same number of azimuth 
angles were required to obtain the four-fold differential 
cross-section. The calculation performed does not display prior-post 
discrepancies at all. Each atomic total cross-section in 
Eq.~(\ref{eq:sumion}) was calculated using 35 to 100 momentum transfer 
values, 28 fixed electron angles, and around 45 electron energies 
depending on the projectile impact energy. Further details of the 
calculation are given in Ref.~\cite{montanari2017}. Simultaneously, we 
will be reporting state to state ionization cross sections for these 
cases in Ref.~\cite{miraglia2019}, which will be very useful to estimate 
molecule fragmentation.

In Figure 1, we report our total ionization results for the six 
essential elements by the impact of six different projectiles: 
antiprotons, H$^{+}$, He$^{2+}$, Be$^{4+}$, C$^{6+}$, and O$^{8+}$. 
To reduce the resulting 36 magnitudes into a single consistent 
figure, we considered the fact that in the first Born approximation
the ionization cross-section scales with the square of the projectile 
charge, $Z^{2}$. The values of the impact energies considered 
range between 0.1 to 10 MeV/amu, where the CDW is supposed 
to hold. In fact, for the highest projectile charges the minimum 
impact energy where the CDW is expected to be valid could be 
higher than 100 KeV. Moreover, we corroborated that the first Born 
approximation provides quite reliable results only for energies higher 
than a couple of MeV/amu. We use the same color to indicate the 
projectile charge throughout the figures of this work: dashed-red, 
solid-red, blue, magenta, olive and orange for antiprotons, 
H$^{+}$, He$^{2+}$, Be$^{4+}$, C$^{6+}$, and O$^{8+}$, respectively. 
Notably, there is no complete tabulation of ionization of atoms by 
the impact of multicharged ions. We hope that the ones presented 
in this article will be of help for future works. 

\section{Ionization of Molecules}

\subsection{The stoichiometric model}

Lets us consider a molecule $M$ composed by $n_{\alpha}$ atoms of the
element $\alpha$, the SSM describes the total ionization cross section 
of the molecule $\sigma_{M}$ as a simple sum of ionization cross 
sections of the isolated atoms $\sigma_{\alpha}$, 
\begin{equation}
 \sigma_{M}=\sum\limits_{\alpha}n_{\alpha}\sigma_{\alpha}\,.  
 \label{eq:sumion}
\end{equation}
We divided sixteen molecular targets of our interest in 3 families. 
The classification defined is given in Table~\ref{tab:families}.
\begin{table}
\begin{center}
\begin{tabular}{|p{0.06\textwidth}|p{0.55\textwidth}|}
\hline
 CH  & CH$_4$, C$_2$H$_2$, C$_2$H$_4$, C$_2$H$_6$, and C$_6$H$_6$ \\
\hline
 CHN & C$_5$H$_5$N, C$_4$H$_4$N$_2$, C$_2$H$_7$N, and CH$_5$N \\
\hline
 \multirow{4}{*}{DNA} & adenine (C$_5$H$_5$N$_5$), 
                        timine (C$_5$H$_6$N$_2$O$_2$), \\
     & cytosine (C$_4$H$_5$N$_2$O), guanine (C$_5$H$_5$N$_5$O), \\
     & uracil (C$_4$H$_4$N$_2$O$_2$), \\
     & DNA backbone (C$_5$H$_{10}$O$_5$P), \\
     & dry DNA (C$_{20}$H$_{27}$N$_7$O$_{13}$P$_2$)\\
\hline
\end{tabular}
\caption{Sixteen molecular targets of our interest classified in 3 
\label{tab:families}
families.}
\end{center}
\end{table}

In Figure 2, we report the total ionization cross sections by the 
impact of multicharged ions for the four fundamental DNA components: 
adenine, timine, cytosine, and guanine. Our reports for uracil and 
DNA backbone containing P are given in Figure 3. For adenine, the 
agreement with the experimental data available in our range of validity 
is quite good. For uracil, we have a puzzling situation: our results 
show good agreement with the experiments by proton impact of Itoh 
{\it et al.}~\cite{itoh2013}, but for the same target our theory fails 
by a factor of two for the impact of C$^{4+}$ and O$^{6+}$ ions. 
Nonetheless, it should be stated that our theoretical results coincide 
with the ones by Champion, Rivarola and 
collaborators~\cite{champion2012,agnihotri2012}, which may indicate a 
possible misstep of the experiments. 

\subsection{Scaling rule}

The first attempt to develop a comprehensive but straightforward 
phenomenological model for electron ejection from large molecules was 
proposed by Toburen {\it et al.}~\cite{toburen1975,toburen1976}. 
The authors found it convenient to scale the experimental ionization 
cross section in terms of the number of outer or weakly bound valence 
electrons (total number of electrons minus the inner-shell ones). 
We introduce the cross section per weakly bound electron or ionization 
cross section per valence electron, $\sigma_{e}$, as
\begin{equation}
\sigma_{e}=\frac{\sigma_{M}}{N_{M}}=\frac{\sum\limits_{\alpha}
n_{\alpha}\sigma_{\alpha}}{\sum\limits_{\alpha}n_{\alpha}\nu_{\alpha}}
=\sigma_{e}(v)\,, 
\label{27} 
\end{equation}
where
\begin{equation}
\nu_{\alpha}=\left\{ 
\begin{array}{ll}
4, & \text{for C,} \\ 
5, & \text{for N and P,} \\ 
6, & \text{for O and S,}
\end{array}\right.
\label{eq:nelec} 
\end{equation}
and $\nu_{H}=1$. The Toburen rule can be stated by saying that 
$\sigma_{e}$ is a \textit{universal} parameter independent on the 
molecule, which depends solely on the impact velocity. 
In Ref.~\cite{toburen1976}, it was found experimentally that for proton 
impact on some simple molecules, $\sigma_{e}$ results
\begin{equation}
\begin{array}{cccc}
E\text{(MeV)}               & 0.25 & 1.00 & 2.00 \\ 
\sigma_{e} (10^{-16}cm^{2}) & 0.39 & 0.17 & 0.11
\end{array}
\label{30}
\end{equation}
with an estimated error of about 20\%. A similar ratio was found in 
Ref.~\cite{itoh2013} for proton impact on uracil and adenine. 
Figure 4a shows $\sigma_{e}$ calculated with the CDW as a function of 
the impact velocity for different projectile charges computed with the 
SSM for the sixteen molecular targets displayed in 
Table~\ref{tab:families}. The universality with $Z$ is the one provided 
by Born approximation, i.e., $\sigma _{e}(Z)=Z^{2}\sigma_{e}(Z=1)$, 
and it holds for large impact velocities, as shown Figure 4a.
Of course, for lower impact velocities, the CDW breaks the behavior of 
the $Z^{2}$ rule. Although the Toburen rule holds for high energies, 
its performance is still not satisfactory: the universal band is quite 
broad. The departure of our theoretical 
results from the Toburen rule can be easily explained by the fact that 
our atomic cross sections $\sigma_{\alpha}$ behave differently for 
each atom in terms of the projectile charges. By inspecting Figure 1, 
one can easily see that the rule $\sigma_{\alpha}/v_{\alpha}\sim
\sigma_{e}$ is not well satisfied by the CDW. A much better general 
atomic rule is given by
\begin{equation}
\sigma_{e}^{\text{CDW}}=\frac{\sum\limits_{\alpha}
n_{\alpha}\sigma_{\alpha}^{\text{CDW}}}{\sum\limits_{\alpha}n_{\alpha}
\nu_{\alpha}^{\text{CDW}}},
\label{32} 
\end{equation}
where 
\begin{equation}
\nu_{\alpha }^{\text{CDW}} \sim\left\{ 
\begin{array}{ll}
4,\ \ \  & \text{for C, N, and O,} \\ 
4.5,\ \  & \text{for P and S}
\end{array}
\right. 
\label{35}
\end{equation}
and, obviously, $\nu_{H}^{\text{CDW}}=1$. The cross sections 
$\sigma_{e}^{\text{CDW}}$ 
are plotted in Figure 4b. A much better sharp band is observed, 
especially at high energy, where the theory is expected to work. It 
will be interesting for experiments to cross-check this theoretical 
prediction.

\subsection{Emitted electron energies}

In a given biological medium, direct ionization by ion impact accounts 
for just a fraction of the overall damage. Secondary electrons, as well 
as recoil target ions, also contribute substantially to the total damage. 
We can consider the single differential cross section of the shell 
$nl$ of the atom $\alpha$, $d\sigma _{\alpha nl}/dE$, to be a function 
of the ejected electron energy $E$ as a simple distribution 
function~\cite{surdutovic2018}. Then, we can define the mean value 
$\overline{E}_{\alpha}$ as in Ref.~\cite{abril2015},
\begin{eqnarray}
\overline{E}_{\alpha} &=&\frac{\langle E_{\alpha}\rangle}{\langle
1\rangle}=\frac{1}{\sigma_{\alpha}}\sum\limits_{nl}\int dE\,E
\frac{d\sigma_{\alpha,nl}}{dE}\,,  
\label{40} \\
\langle 1\rangle &=&\sigma_{\alpha}=\sum\limits_{nl}\int dE\,
\frac{d\sigma_{\alpha,nl}}{dE}\,,  
\label{50}
\end{eqnarray}
where $\Sigma_{nl}$ takes into account the sum of the different 
sub-shell contributions of the element $\alpha$.

Figure 5 shows $\overline{E}_{i}$ for six elements from 
Table~\ref{tab:families}. The range of impact velocities was shorten up 
to $v=10$ a.u. due to numerical limitations in the spherical harmonics 
expansion. In our theoretical treatment, we expand our final continuum 
wave function as per usual,
\begin{equation}
\psi_{\overrightarrow{k}}^{-}(\overrightarrow{r})=\sum_{l=0}^{l_{\max
}}\sum_{m=-l}^{l}R_{kl}^{-}(r)Y_{l}^{m}(\widehat{r})Y_{l}^{m^{\ast }}
(\widehat{k})\,.
\label{60}
\end{equation}
We are confident with our calculations up to $l_{\max}\sim 30$. 
As the impact velocity $v$ increases, so do $\langle E_{\alpha}\rangle$
and $l_{\max}$, which results in the inclusion of very oscillatory 
functions in the integrand. Furthermore, the integrand of
$\langle E_{\alpha}\rangle$ includes the kinetic energy $E$
(see Eq.~(\ref{40})), which cancels the small energy region and 
reinforces the large values, making the result more sensible to large
angular momenta. Regardless, for $v>10$ a.u., the first Born 
approximation holds.

In Figure 5, we estimate $\overline{E}_{\alpha}$ in the 0.5--2 a.u.
velocity range, or equivalently from 15 to 50 eV, for all the targets.
Our results agree with the experimental findings~\cite{surdutovic2018}. 
The dependence of the mean energy value with the projectile charge $Z$ 
is surprisingly sensible, which can duplicate the proton results. 
This effect can be attributed to the depletion caused by the 
multicharged ions to the yields of low energy electrons. In the high 
electron energy regime, the CDW falls on the simple first Born 
approximation, surviving the $Z^{2}$ law. Then, the ratio in 
Eq.~(\ref{40}) cancels out and $\overline{E}_{\alpha}$ becomes a 
universal value independent on $Z$.

Extending the simple stoichiometric model for the mean electron energy
calculation, it results
\begin{equation}
\overline{E}_{M}=\frac{\sum\limits_{\alpha}n_{\alpha}
\overline{E_{\alpha}}}{\sum\limits_{\alpha}n_{\alpha}\sigma_{\alpha}}.
\label{70}
\end{equation}
For impact of H$^{+}$ and He$^{2+~}$ on water at 1 MeV/amu, we obtain 
$\overline{E}_{H_{2}O}=43.7$ and 45 eV, while the experimental values 
on liquid water by Pimblott and LaVerne [xxx] were found to be 51.5 and 
52.2 eV, respectively. Just to stress the importance of the projectile 
charge, for O$^{8+}$ on water, at the same impact energy 1 MeV/amu, 
we obtained $\overline{E}_{H_{2}O}=54.7$ eV, which is 25\% larger than 
proton impact result. It is worth noting that there should not be any 
difference if the calculations were carried in first Born approximation. 
We will came back on this issue later.

\subsection{Emitted electron angles}

As mentioned before, secondary electrons contribute to the total damage. 
Then, it is important not only the ejection energy but also the angle 
of emission. Once again, we can consider the single differential cross 
section in terms of the ejected electron solid angle $\Omega$, 
$d\sigma_{\alpha,nl}/d\Omega$, to be expressed as a distribution 
function, and define the mean angle $\overline{\theta}_{\alpha}$ as
\begin{equation}
\overline{\theta}_{\alpha}=\frac{\langle\theta_{\alpha}\rangle}
{\langle 1\rangle}=\frac{1}{\sigma_{\alpha}}\sum\limits_{nl}
\int d\Omega\,\theta\,\frac{d\sigma_{\alpha,nl}}{d\Omega}
\end{equation}
Figure 6 shows $\overline{\theta}_{\alpha}$ for our six elements of 
interest. A new important dependence of $\overline{\theta}_{\alpha}$ 
with $Z$ is observed. It is the common view~\cite{Rudd1992} that the 
angular dispersion of emitted electrons are nearly isotropic due to the
fact that angular anisotropy of sub-50-eV yield is not significant. 
A typical value of ejection angle used in the literature is 
$\overline{\theta}_{\alpha}\sim 70$ degrees~\cite{surdutovic2018}, and 
it is quite correct in the range of validity of the first Born 
approximation for any target. But when a distorted wave approximation 
is used, $\overline{\theta}_{\alpha}$ decreases substantially with $Z$ 
in the intermediate energy region, as observed in Figure 6. 
For example, for C$^{6+}$ impact, the Bragg peak occurs at 0.3 MeV/amu, 
where $\overline{\theta}_{\alpha}$ computed with the CDW method is 
about half of the value obtained with the first Born approximation. 
This correction should close the damage to the forward direction.

The only responsible of this correction is the capture to the continuum
effect: the larger the charge $Z$ the smaller $\overline{\theta}$ will 
be, of course at intermediate energies, not at high impact energies, 
where the Born approximation rules. One illustrative observation is the 
behavior of antiprotons: the projectile in this case repels the 
electrons making the distribution almost symmetric. Note the opposite 
effect of proton and antiprotons, they run one opposite to the other, 
as compared with the first Born approximation.

\subsection{A modified stoichiometric model}

One can correctly argue that\ the SSM considers an assembly of isolated
neutral atoms which is definitively unrealistic. A first improvement is to
consider that the atoms are not neutral and within the molecule they have
more or less electrons given by a \ charge $q_{\alpha }$. One parameter that
measures $q_{\alpha }$ is the Mulliken charge. This is not the only one,
there is a large variety of charges such as the net or natural atomic charge 
\cite{lee2003}, the lowin charge, etc.

Consider\ that the total amount of electrons $Q_{\alpha }\ $on element $%
\alpha $ are equally distributed on all the atoms $\alpha ,$ therefore each
element $\alpha \ $\ will have a an additional charge: $q_{\alpha
}=Q_{\alpha }/n_{\alpha }$. positive or negative depending on the relative
electronegative value respect to the other ones~\cite{rappe1991}. Instead
of the integer number of elements $n_{\alpha }$ of the atom $\alpha$, it
could be argued that we have now an continuous number of atoms given by 
\begin{equation}
n_{\alpha }^{\prime }=n_{\alpha }-\frac{q_{\alpha }}{v_{\alpha }^{CDW}}
\label{100}
\end{equation}%
In the case of neutral atoms $q_{\alpha }=0,$ we obviously recover $%
n_{\alpha }^{\prime }=n_{\alpha }$ as before$.$

To inspect the effect of the $q_{\alpha }$ we have run GAMES with the 
6-31G basis set, including polarization functions for all the atoms. 
The calculations were carried out implementing the B3LYP 
functional~\cite{Becke1993,Stephens1994}. In Table~(\ref{110}) we display
$q_{\alpha }$ of the different DNA basis and in the last column we rewrite
the new stoichiometric expression. 
\begin{equation}
\begin{array}{|l|l|l|l|l|l|}
\hline
\text{Element}\,\,\, & \mathrm{H} & \mathrm{C} & \mathrm{N} & \mathrm{O} & 
\text{New stoichiometry} \\ 
\hline
\mathrm{Adenine}  & +0.32 & -0.55 & +0.23 &       & 
\mathrm{C}_{4.92}\mathrm{H}_{5.14}\mathrm{N}_{4.77} \\ 
\hline
\mathrm{Thymine}  & +0.20 & -0.54 & +0.19 & -0.52 & 
\mathrm{C}_{4.95}\mathrm{H}_{6.13}\mathrm{N}_{1.95}\mathrm{O}_{2.13} \\ 
\hline
\mathrm{Cytosine} & +0.28 & -0.56 & +0.21 & -0.53 & 
\mathrm{C}_{3.93}\mathrm{H}_{5.14}\mathrm{N}_{2.79}\mathrm{O}_{1.13} \\ 
\hline
\mathrm{Guanine}  & +0.46 & -0.58 & +0.20 & -0.36 & 
\mathrm{C}_{4.89}\mathrm{H}_{5.15}\mathrm{N}_{4.80}\mathrm{O}_{1.09} \\ 
\hline
\end{array}
\label{110}
\end{equation}

All the previous formula holds with the simple replacement of integer value $%
n_{\alpha }$ by the continuous values\ $n_{\alpha }^{\prime }$ to produce a
new ionization cross section $\sigma ^{\prime }$

In Figure 7, we displays the relative difference of ionization cross
sections 
\begin{equation}
e=\frac{\sigma ^{\prime }-\sigma }{\sigma }\times 100  \label{120}
\end{equation}%
for the four DNA basis. The difference is \ just very few percents.
Indicating that the SSM is a quite robust model to treat the type of
molecule we are dealing with, and of course within the range error that one
can expect from the SSM.

We now return to the calculation of the mean electron energy of H$^{+}$ and
He$^{2+~}$ in water\ \ $\overline{E}_{H_{2}O}$. If we consider that \ the
residual charge of \ each hydrogen in water is $q_{H}=$+0.35 \ \cite%
{rappe1991}\ which implies that on Oxygen we have \ a residual charge of $%
q_{O}=$-0.7, we could re-write the formula of water using (\ref{120}) as H$%
_{1.3}$O$_{1.17}$ considering $n_{O}^{CDW}\simeq 4$ and $n_{H}^{CDW}\simeq
1. $ Recalculating mean electron energy by impact of \ H$^{+}$ and He$^{2+~}$
on water at 1 MeV/amu we obtain: $\overline{E}_{H_{2}O}=$46.2 \ and 47.6 eV
which compared a little better with the experiments: 51.5 and 52.2 eV \cite%
{pimblott2007}, respectively. i.e. within 10\% of the experiments which is
not bad accounting for the simplicity of our method.

Another parameter that we can calculate in our model is the fraction of
energy carried by the electron $f$ defined as%
\begin{equation}
f=\frac{\sum\limits_{\alpha }n_{\alpha }\sigma _{\alpha }\overline{%
E_{\alpha }}}{\sum\limits_{\alpha }n_{\alpha }(\sigma _{\alpha }\overline{%
E_{\alpha }}+\sum\limits_{nl}\sigma _{\alpha nl}E_{\alpha nl})}  \label{90}
\end{equation}%
Using our shell-to-shell ionization cross sections $\sigma _{\alpha nl}\ $%
\cite{miraglia2019}\ and atomic binding energies of the atom$\ \alpha ,$ $%
E_{\alpha nl}\ $\cite{clementi}, we obtain $f=$0$.70\ $and $\ 0.71$ for \ H$%
^{+}$ and He$^{2+~}$ on water at 1 MeV/amu,. which differ from the
experiments \ $f=0.81$ and $0.80$, \ respectively.

This difference with the experiment could be attributed to the difference of
binding energy between the molecular water 12.65 eV and the atomic component
H (13.6 eV) and O (17.19 eV), which is an issue beyond our independent atom
approximation and so our model cannot not accounted for. We could estimate
this contribution resorting, once more to the stoichiometric, model, we can
arriesgar a upper-limit difference per molecule of water\ as

\begin{equation}
\delta E=(13.6-12.65)1.17+(17.19-12.65)1.3=7\ \text{eV}  \label{80}
\end{equation}%
Most of this energy should be transfer to the kinetics energy of the emitted
electron. Thus we can \ add 7 eV to the previous values to give $\overline{E}%
_{H_{2}O}=$53.2 \ and 54.6 eV for \ H$^{+}$ and He$^{2+~}$ on water at 1
MeV/amu, respectively, much near to the experiments: 51.5 and 52.2 eV. If we
add $\delta E$ to the numerator of Eq.(\ref{90}) we obtain\ $f=0.81$ and $%
0.80$ in perfect agreement with the experiments (see Table 1 of Ref.(\cite%
{pimblott2007})

\section{CONCLUSIONS}

We have calculated ionization cross sections by impact of antiprotons, H$^{+}
$, He$^{2+}$, Be$^{4+}$, C$^{6+}$,\ and O$^{8+}$ with molecule involved in
biological basis containing H, C, N, O, P and S with the CDW method. \ The
importance of the influence of $Z\ $was observed in the mean energy $%
\overline{E}_{\alpha }\ $and angle $\overline{\theta }_{\alpha }.$ For a
given target $\alpha ,$ as $Z$ increases $\overline{E}_{\alpha }$ increases
but $\overline{\theta }_{\alpha }$ decreases \ At high impact energy, say
larger than 1 MeV/amu these values tend to the ones of the Born
approximation which embodies the simple Z$^{2}$ law. Sixteen molecules were
investigated using the simple stoichiometric model. Results for the six DNA
basis were presented and compared with the sparse available experiments \ We
explore the rule of Toburen which scales all the molecular ionization cross
section when divided by the number of weakly bound valence electrons $\nu
_{\alpha }$ given by Eq.(\ref{27})$.$ We have found the rule scales much
better when normalizing our theoretical ionization cross sections to the
number\ $\upsilon _{\alpha }^{\prime }$ given by Eq.(\ref{35}). And finally
we attempt to improve the stoichiometric model by the use of the Mulliken
charge to redefine a new stoichiometric model containing continuum rather
than integer proportions. NO substantial correction was found indicating
that the SSM works quite well.

Our aims is this article is to provide the tools to estimate any inelastic
parameter parameter -such as the emission angle, the emitted mean energy and
cross section- by the impact of any multicharged \ on any molecule
containing H, C, N, O, P and S, with the help of the stoichimetrical model.
Our goal was quite \textbf{pretencioso}, considering the simplicity of our
proposal. However we think our results could be used to estimate the
ionization magnitude with an acceptable level of uncertailties

\bigskip

\section{References}

\begin{thebibliography}{99}

\bibitem{gamess}
M. W. Schmidt, K. K. Baldridge, J. A. Boatz, S. T. Elbert, M. S. Gordon, 
J. H. Jensen, S. Koseki, N. Matsunaga, K. A. Nguyen, S. J. Su, T. L. Windus, 
M. Dupuis, J. A. Montgomery 
J. Comput. Chem. 14, 1347-1363 (1993)

\bibitem{miraglia2008} 
J. E. Miraglia and M. S. Gravielle. Ionization of the
He, Ne, Ar, Kr, and Xe isoelectronic series by proton impact. Phys Rev A 
\textbf{78}, 052705 (2008)

\bibitem{miraglia2009} 
J. E. Miraglia, Ionization of He, Ne, Ar, Kr, and Xe
by proton impact: Single differential distributions. Phys Rev A \textbf{79},
022708 (2009).

\bibitem{mendez2016} 
A.M.P. Mendez, D.M. Mitnik, and J.E. Miraglia.
Depurated inversion method for orbital-specific exchange potentials. 
Int. J. Quantum Chem. 24 ,116 (2016).

\bibitem{mendez2018} 
A.M.P. Mendez, D.M. Mitnik, and J.E. Miraglia. Local Effective 
Hartree--Fock Potentials Obtained by the Depurated Inversion Method,
76. (2018).

\bibitem{montanari2017} 
Ionization probabilities of Ne, Ar, Kr, and Xe by
proton impact for different initial states and impact energies. Nucl. Instr.
Meth. Phys. Res. B 407 (2017) 236-243.

\bibitem{miraglia2019} 
J. E. Miraglia. Shell-to-shell ionization cross
sections of antiprotons, H$^{+}$, \ He$^{2+},$ Be$^{4+},$ C$^{6+}$ and O$%
^{8+}$ on H, C, N, O, P, and S atoms To be published Archive 2019.

\bibitem{itoh2013} 
A. Itoh, Y. Iriki, M. Imai, C. Champion, and R. D.
Rivarola,~Cross sections for ionization of uracil by MeV-energy-proton
impact,\ Physical Review A 88, 052711 (2013)

\bibitem{} in energy and angles~,

\bibitem{champion2012} 
C Champion, M E Galassi, O Foj\'{o}n, H Lekadir, J Hanssen, R D Rivarola,
P F Weck, A N Agnihotri, S Nandi, and L C Tribedi. Ionization of RNA-uracil
by highly charged carbon ions. J. Phys.: Conf. Ser. 373, 012004 (2012).

\bibitem{agnihotri2012}
A. N. Agnihotri, S. Kasthurirangan, S. Nandi, A.
Kumar, M. E. Galassi, R. D. Rivarola, O. Foj\'{o}n, C. Champion, J. Hanssen,
H. Lekadir, P. F. Weck, and L. C. Tribedi. Ionization of uracil in
collisions with highly charged carbon and oxygen ions of energy 100 keV to
78 MeV.\ \ Phys Rev A 85, 032711 (2012).

\bibitem{toburen1975} 
W. E. Wilson and L. H. Toburen. Electron emission from
proton ---hydrocarbon-molecule collisions at 0.3---2.0 MeV. PHYSICAL REVIEW
A\ 11 1303 (1975).

\bibitem{toburen1976} 
D. J. Lynch, L. H. Toburen, and W. E. Wilson. Electron
emission from methane, ammonia, monomethylamine, and dimethylamine by 0.25
to 2.0 MeV protons. J. Chem. Phys. 64, 2616 (1976).

\bibitem{surdutovic2018} 
Multiscale approach to the physics of radiation
damage with ions. E. Surdutovich and A. V. Solov'yov, arXiv:1312.0897v,
(2013)

\bibitem{abril2015} 
P. de Vera1, I. Abril, R. Garcia-Molina and
A.V.Solov'yov,\ Ionization of biomolecular targets by ion impact: input data
for radiobiological applications. Journal of Physics: Conference Series 438
(2013) 012015

\bibitem{lee2003} 
Jung-Goo Lee, Ho Young Jeong, and Hosull Lee, Charges of
Large Molecules Using Reassociation of Fragments. Bull. Korean Chem. Soc.24\
2003, 369 .

\bibitem{rappe1991} 
A. K. Rappe, A. K.and W. A. Goddard III,. J. Phys. Chem. 
\textbf{95 (}1991) 3358.

\bibitem{Becke1993}
A. D. Becke, 
J. Chem. Phys. 98, 5648-5652 (1993) 

\bibitem{Stephens1994}
P. J. Stephens, F. J. Devlin, C. F. Chabalowski, M. J. Frisch,
J. Phys. Chem. 98, 11623-11627 (1994) 

\bibitem{pimblott2007} 
S.M. Pimblott and J. A. LaVerne, Radiation
Physics and Chemistry 76, 1244-1247 (2007)

\bibitem{Rudd1992} 
M. E. Rudd, Y.-K. Kim,, D. H. Madison and T. J. Gay.
Electron production in proton collisions with atoms and molecules: energy
distributions. Rev. Mod. Phys. \textbf{64}, 44-490 (1992).

\end{thebibliography}

\end{document}
