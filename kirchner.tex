Dear Claudia,

thanks for your email and the preprint. Good to know that you have
started to work on this as well! In fact, we found your preprint on
arXiv (via researchgate) a few days ago and discussed it a bit already.
So, if you don't mind, here are a few comments/questions:

(i) It's gratifying to see that your scaling (Eq. (9)) agrees with our
findings for C,N,O, but this raises the questions why we disagree on P.
Would it be worthwhile to compare the atomic ionization cross sections
for proton impact? Perhaps we should begin with a sanity check. We are
using Talman's OPM (or OEP) which results in -18.9 and -10.5 eV for the
eigenvalues of P(3s) and P(3p). What do you have?

(ii) BTW, we have another paper on this topic in the pipeline (it's with
EPJD). I attach the revised version of it which we prepared in response
to the recommendations of the reviewers in the first round of
refereeing. The paper focuses a bit more on the nucleobases and compares
our IAM-PCM to the CNDO approach used by Roberto and his co-workers.

(iii) This brings me to the model that you propose in Sec III C of your
manuscript. Can you comment on its relation to the CNDO approach? It
seems to be similar in spirit (because it is based on a Mulliken
analysis), but simplified/modified. I say simplified because the CNDO
approach uses different Mulliken 'gross populations' for the
contributing AOs whereas your model is based on the charges obtained
from summing up the AO contributions. I say modified because of Eq.
(10). Do you agree with this?

(iv) About Eq. (10): I would have expected to find the big Q_alpha in
the numerator of the second term on the RHS, not the little one. Your
refer to Q_alpha as the 'total amount of electrons' found on the atoms
of a certain species. Dividing this number by your eta_alpha^CDW (the
denominator) - i.e., the effective number of electrons available for
ionization in one free atom of that kind - would give you the additional
fractional number of atoms of that species in the molecule ('additional'
refers to a negative charge Q_alpha). That's what I would add to
n_alpha. Or am I mistaken?

(v) If I use the big Q_alpha instead the little one and the Mulliken
charges provided in Table III of your manuscript I obtain for uracil the
stoichiometric coefficients 3.69, 3.12, 2.30, 2.24 instead of 3.92,
3.78, 2.15, 2.12 (there is clearly a typo in that column of your table -
you have 1.78 for H and 4.15 for N, but you should have 3.78 and 2.15).
Looks like a bigger effect, but when I sum everything up the overall
effect will be (even) smaller, because the positive and negative
corrections cancel almost completely. That's just an observation. I
haven't done this for the other molecules.

(vi) Regardless of (iii) to (v) it seems to me that the overall effect
of any Mulliken or CNDO variant is small compared to what you call
'simple SSM' (which is what we refer to as IAM-AR). That's consistent
with our own findings and, I believe, with Roberto's findings as well.
By contrast, our IAM-PCM can result in a substantial reduction of the
IAM-AR cross sections at low and intermediate energies. This effect is
most dramatic for electron capture at low energies (as shown in the
attached preprint).

(vii) We have also done calculations for multiply-charged ion impact and
are in the process of analyzing them. First results indicate that we
obtain good agreement with Lokesh's measurements for uracil at MeV
energies using our IAM-PCM. So we might disagree with your comment that
there might be a 'possible misstep of the experiments', but we still
have to complete our analysis.

Best regards,

Tom

