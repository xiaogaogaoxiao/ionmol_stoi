\documentclass[a4paper,12pt]{article}
\usepackage[utf8]{inputenc}
\setlength{\parindent}{0pt}
\usepackage{soul}
\usepackage{xcolor}
\usepackage[margin=0.8in]{geometry}
\usepackage{graphicx}

\setstcolor{blue}
\def\reviewer#1{\vspace{0.35cm}\textsl{#1}}
\def\reply#1{\vspace{0.1cm}\textcolor{blue}{#1}}

\title{Response to referee reports \\ JPHYSB-106014}

\begin{document}
\maketitle

\reply{We appreciate the comments and questions of the three referees. 
They were taken into account, as explained in the replies to each 
referee. In what follows, the referee reports are in black letters and 
our answers and comments in blue. }

\reply{\textbf{Summary of the modifications:} The manuscript was 
changed from letter to regular manuscript by the editorial board. 
Attending to this fact and the comments by the three referees 
requesting further explanations, we decided to modify the manuscript by
dividing the text into three sections: Introduction, Scaling rules (with
subsections), and Conclusions. Additional specifications of the scaling 
models are given in Section II-A, B, and C. Figure 2 was improved by 
including a gray area of $\pm 30\%$ that covers most of the available 
data. Also, experiments for H$^+$ and He$^{+2}$ in water (missing in 
the previous version) were added to the figures (Refs. 25, 26, and 27), 
extending the energy range where the present scaling agrees with the 
experimental data. 
% the agreement of the present scaling for H$^+$ in water above 40 keV 
% and He$^{+2}$ in water up to 10 MeV/amu. 
Following the suggestion by one of the referees, we changed the title 
from ``Universal scaling for the ionization...'' to ``Scaling 
rules for the ionization...''}.

% ----------------------------------------------------------------------
\section{Reply to report of the First Referee}
% ----------------------------------------------------------------------

\reviewer{The authors propose the universal scaling law of the 
ionization cross sections by the impact of multiply charged atomic ions 
on the basis of the experimental data reported in their very recent 
publication [J. Phys. B, 53, 055201 (2020)]. Their conclusion of this 
newly submitted paper is that they found the scaling law of Eq. (1) 
with the alpha value of 1.2. However, as the authors mention in the 
introduction part, the same scaling law with the alpha value of 
4/3 = 1.333 was reported in 2013 by DuBois et al. [Ref. 15]. The 
authors should explain if the scaling with alpha = 4/3, which is very 
close to their value of 1.2, could not explain the experimental data 
they recorded. If the value of 1.2 is much better than 1.333, the 
author should explain possible reason why alpha should be 1.2 on the 
basis of the physical meaning of the parameter alpha.}

\vspace{0.2cm}

\reply{The value of $\alpha=1.2$ was found numerically as the best
convergence of the CDW-SSM cross sections of the forty collisional 
systems over the broadest energy range. Our methodology does not rely 
on the fitting of experimental data. The work considers the scaling of 
our theoretical results, which is tested afterward with the 
experimental data. We also noticed that all the maxima of the 
theoretical cross sections scaled with $\alpha=1.2$ lay at the same 
value of $E/Z^{2-\alpha}=E/Z^{0.8}$. In the present version of our 
manuscript, we modified lines 59-61 to clarify our choice of $\alpha$.}

\reply{About the  physical meaning of this parameter, the scaling by
Janev et~al.~[14] uses $E/Z^1$ as abscissa based on the long-distance
approximation for the ionization cross sections. Similarly, DuBois 
et~al.~[15] propose $E/Z^{2/3}$. The 2/3 in Ref.~[15] was found by 
trial and error. However, the dependence with $Z^{2/3}$ recalls the 
Thomas Fermi model for a constant density of electrons around the ion. 
% Also in [15], it is mentioned that $E/Z^{0.88}$ joins  the maximum of 
% the cross sections of He and H$_2$. 
In either case, both models are approximations, and our numerical 
proposal of $\alpha$ is located between them.} 

\vspace{0.2 cm}
% ----------------------------------------------------------------------
\section{Reply to report of the Second Referee}

The authors propose a scaling formula for cross sections for ionization
of complex molecules by multiply charged ions. Such cross sections are
required, e.g., for estimating radiation damage in biomolecules. A
routine ab-initio calculation of such cross sections by quantum methods
is beyond reach of the state-of-the-arts methods. In this situation, 
it is helpful to have semi-empirical scaling formulas for modeling
purposes. Thus, the paper is timely and serves a need. It is concise.
However, there are some language issues. The paper should eventually be
published as a JPB letter after the authors having adequately responded
to my detailed points below. 

% ----------------------------------------------------------------------

\vspace{0.2cm}

\reply{The points 1)-4), 6) and 7) raised by the referee are language
corrections. They were all considered and modified in the manuscript
where appropriate.}

%\reviewer{1) page1, line 11: `` ... from +1 to +8 colliding with five ...''}
%\reviewer{2) page 1, line 14: `` ... fourty ...''}
%\reviewer{3) page 1, line 26: `` ... continuum distorted-wave ...''}
%\reviewer{4) page 1, line 42: ``... first order approximation in experimental ...''}
%\reviewer{6) page 1, line 20, column 2: ``... fourty ...''}
%\reviewer{7) page 1, line 24,, column 2: `` ... and five ion species: ...''}

\reviewer{5) page 1, line 53: I do not see how any value for alpha keeps
the $Z^2/E$ relationship? $\alpha = 1.2$ results in $Z^{0.8}/E$.}

\reply{In first order approaches, the ionization cross section $\sigma$
depends linearly with $Z^2/E$. This dependence is valid in the first Born 
approximation, and at high impact energies, $\sigma/Z^{2}$ scales with
$E$ for any charged ion. Within this approximation, $\sigma/Z^{\alpha}$
will be a function of $Z^{2-\alpha}/E$ for any value of $\alpha$. At
intermediate energy, the linear dependence with $Z^2/E$ no longer holds,
so we numerically sought for the value of $\alpha$ that best merges
the cross sections for different impinging charges.}

\reviewer{8) Figure captions. Both caption should mention that all 
plots are for alpha=1.2.}

\reply{Done.}

\reviewer{9) page 2, line 40, column 2: How broad is the ``narrow 
band''? The authors should supply some estimate for the uncertainty of 
the scaling formula. I consider this very important.}

\reply{We estimated the uncertainty of the scaling formula to 20\% for 
the theoretical curves of the different targets, and 30\% to incorporate 
most of the available data. The gray area drawn in Fig.~2 accounts for 
this uncertainty. We also included a sentence to refer to this subject 
in lines 119-122.}

\vspace{0.2 cm}
% ----------------------------------------------------------------------
\section{Reply to report of the Third Referee}
% ----------------------------------------------------------------------

This paper tries to obtain a scaling law for ionization of some 
biological molecules by highly charged ions. As is well known, such
processes are quite relevant and important in medical physics. Since 
the molecules considered are complex, formal theories of ionization are
difficult to apply and scaling laws, if found applicable, greatly 
simplify the process of obtaining reliable cross sections. In this
respect the present paper is quite significant. The authors have
proposed a scaling law and have applied it to several bio molecules 
and have tried to extend and justify the validity of the scaling law 
by including results for N2, O2, CO, CO2 and CH4 to show they have the
same scaling behaviour. However, there are several aspects of the paper
which I either do not understand clearly, or have reservations.

\reviewer{1. The authors mention $\sigma$ to be a function of $E/Z$. 
But the functional form is nowhere mentioned in the paper. What is the
functional dependence on $E/Z$ the authors consider in the work?}

\reply{In different energy regions, the ionization cross section 
follows different $Z/E$ laws. At high energies, it is a function of 
$Z^2/E$, as predicted by the first Born approximation. The CDW 
approximation has a more complex dependence with $Z/E$ but tends to the
Born approximation at high energies, as shown in our previous work [8]. 
The objective of this manuscript is precisely to deepen into the study 
of the dependence of the ionization cross sections with $Z$ and $E$ in 
the intermediate energy region. Janev and Presnyakov [14], in their 
paper from 1980, suggest that $\sigma/Z$ is a linear function of $E/Z$ 
at intermediate energies (around the maximum of the cross sections).} 

\reply{In this version, we further explained the models and cross 
section dependence with $E/Z$ (line 24 and the second paragraph of 
Section II A).}

\reviewer{2. I have reservations in calling the scaling law developed 
a universal scaling law. Not only have the authors considered only 8
molecular targets and five incident ions, the results, as the authors
claim, are applicable only around the maximum of the cross sections. 
Though this may be reasonable for practical purposes, in my opinion, 
the results are in too narrow range for calling it a universal scaling
law. I would recommend dropping the word universal everywhere in the
paper.}

\reply{We called our scaling ``universal'' in the sense that it can be
implemented for any ion-target combination within the intermediate 
energy range, i.e., $E/Z^{0.8} \simeq (0.04-5)$ MeV/amu. Nevertheless, 
we followed the referee's recommendation and dropped the word 
``universal'' from the title and the rest of the manuscript. Instead, 
we emphasized the independent nature of the reduced scaling law for 
different ions and molecular targets. We also remaked that even 
the experimental data (in the energy range where our theoretical method
starts to fail) follows the scaling proposed. In that sense, we changed
the last sentence of the abstract, Section II C (lines 127-130, 136),
and the Conclusions (lines 140-154, 157-160).}

\reviewer{3. Figure 1 somehow needs to be improved or made larger,
because it is very difficult to gauge the deviation of the experimental
points from the curves. Is it possible to separate the figure in to two
figures (eg. top four Figure 1, bottom four Figure 2)?}

\reply{Separating Fig.~1 would complicate the ratio of the figure; 
instead, we considered convenient to make the figure larger to utilize 
the space better. We have also reduced the scale of the y-axis of the 
subplots to improve the visualization of the data.}

\reviewer{4. Figure captions need to be significantly improved as it is
difficult to read and correlate the legend descriptions with those in
the figure. For the caption in Figure 2, it is not clear what the
CDW-SSM theoretical results represent ? What is the target molecule
considered?}

\reply{We believe the confusion about Fig.~2 is originated from 
inconsistencies in the text. Figure~2 shows a scaling that is 
independent of the ion charge of the projectile and the molecular 
target. All the figures have the same CDW-SSM results as inputs. We 
improved the explanation about our calculations by adding the last 
paragraph of Section II B (and Eq. (2)), and the first sentence of 
Section II. C. The notation of Eq. (3) was slightly changed to underline 
that the starting point is the CDW-SSM cross sections for molecules.}

%\reviewer{5. The language should be considerably improved as there are many unusual statements.}

%\reviewer{6. There are many typographical errors.}

\vspace{0.5cm}
\reply{Regarding points 5) and 6), we revised our manuscript and we 
believe it has been considerably improved.}

\end{document}
