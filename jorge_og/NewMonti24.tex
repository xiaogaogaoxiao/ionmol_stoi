%\documentclass[pra,showpacs,twocolumn]{revtex4}


\documentclass[preprint,showpacs,pra]{revtex4}
%%%%%%%%%%%%%%%%%%%%%%%%%%%%%%%%%%%%%%%%%%%%%%%%%%%%%%%%%%%%%%%%%%%%%%%%%%%%%%%%%%%%%%%%%%%%%%%%%%%%%%%%%%%%%%%%%%%%%%%%%%%%%%%%%%%%%%%%%%%%%%%%%%%%%%%%%%%%%%%%%%%%%%%%%%%%%%%%%%%%%%%%%%%%%%%%%%%%%%%%%%%%%%%%%%%%%%%%%%%%%%%%%%%%%%%%%%%%%%%%%%%%%%%%%%%%
\usepackage{amssymb}
\usepackage{amsmath}
\usepackage{graphicx}

\setcounter{MaxMatrixCols}{10}
%TCIDATA{OutputFilter=LATEX.DLL}
%TCIDATA{Version=5.50.0.2953}
%TCIDATA{<META NAME="SaveForMode" CONTENT="1">}
%TCIDATA{BibliographyScheme=Manual}
%TCIDATA{LastRevised=Thursday, March 07, 2019 14:11:39}
%TCIDATA{<META NAME="GraphicsSave" CONTENT="32">}
%TCIDATA{Language=American English}

% Macros for Scientific Word 2.5 documents saved with the LaTeX filter.
%Copyright (C) 1994-95 TCI Software Research, Inc.
\typeout{TCILATEX Macros for Scientific Word 2.5 <22 Dec 95>.}
\typeout{NOTICE:  This macro file is NOT proprietary and may be 
freely copied and distributed.}
%
\makeatletter
%
%%%%%%%%%%%%%%%%%%%%%%
% macros for time
\newcount\@hour\newcount\@minute\chardef\@x10\chardef\@xv60
\def\tcitime{
\def\@time{%
  \@minute\time\@hour\@minute\divide\@hour\@xv
  \ifnum\@hour<\@x 0\fi\the\@hour:%
  \multiply\@hour\@xv\advance\@minute-\@hour
  \ifnum\@minute<\@x 0\fi\the\@minute
  }}%

%%%%%%%%%%%%%%%%%%%%%%
% macro for hyperref
\@ifundefined{hyperref}{\def\hyperref#1#2#3#4{#2\ref{#4}#3}}{}

% macro for external program call
\@ifundefined{qExtProgCall}{\def\qExtProgCall#1#2#3#4#5#6{\relax}}{}
%%%%%%%%%%%%%%%%%%%%%%
%
% macros for graphics
%
\def\FILENAME#1{#1}%
%
\def\QCTOpt[#1]#2{%
  \def\QCTOptB{#1}
  \def\QCTOptA{#2}
}
\def\QCTNOpt#1{%
  \def\QCTOptA{#1}
  \let\QCTOptB\empty
}
\def\Qct{%
  \@ifnextchar[{%
    \QCTOpt}{\QCTNOpt}
}
\def\QCBOpt[#1]#2{%
  \def\QCBOptB{#1}
  \def\QCBOptA{#2}
}
\def\QCBNOpt#1{%
  \def\QCBOptA{#1}
  \let\QCBOptB\empty
}
\def\Qcb{%
  \@ifnextchar[{%
    \QCBOpt}{\QCBNOpt}
}
\def\PrepCapArgs{%
  \ifx\QCBOptA\empty
    \ifx\QCTOptA\empty
      {}%
    \else
      \ifx\QCTOptB\empty
        {\QCTOptA}%
      \else
        [\QCTOptB]{\QCTOptA}%
      \fi
    \fi
  \else
    \ifx\QCBOptA\empty
      {}%
    \else
      \ifx\QCBOptB\empty
        {\QCBOptA}%
      \else
        [\QCBOptB]{\QCBOptA}%
      \fi
    \fi
  \fi
}
\newcount\GRAPHICSTYPE
%\GRAPHICSTYPE 0 is for TurboTeX
%\GRAPHICSTYPE 1 is for DVIWindo (PostScript)
%%%(removed)%\GRAPHICSTYPE 2 is for psfig (PostScript)
\GRAPHICSTYPE=\z@
\def\GRAPHICSPS#1{%
 \ifcase\GRAPHICSTYPE%\GRAPHICSTYPE=0
   \special{ps: #1}%
 \or%\GRAPHICSTYPE=1
   \special{language "PS", include "#1"}%
%%%\or%\GRAPHICSTYPE=2
%%%  #1%
 \fi
}%
%
\def\GRAPHICSHP#1{\special{include #1}}%
%
% \graffile{ body }                                  %#1
%          { contentswidth (scalar)  }               %#2
%          { contentsheight (scalar) }               %#3
%          { vertical shift when in-line (scalar) }  %#4
\def\graffile#1#2#3#4{%
%%% \ifnum\GRAPHICSTYPE=\tw@
%%%  %Following if using psfig
%%%  \@ifundefined{psfig}{\input psfig.tex}{}%
%%%  \psfig{file=#1, height=#3, width=#2}%
%%% \else
  %Following for all others
  % JCS - added BOXTHEFRAME, see below
    \leavevmode
    \raise -#4 \BOXTHEFRAME{%
        \hbox to #2{\raise #3\hbox to #2{\null #1\hfil}}}%
}%
%
% A box for drafts
\def\draftbox#1#2#3#4{%
 \leavevmode\raise -#4 \hbox{%
  \frame{\rlap{\protect\tiny #1}\hbox to #2%
   {\vrule height#3 width\z@ depth\z@\hfil}%
  }%
 }%
}%
%
\newcount\draft
\draft=\z@
\let\nographics=\draft
\newif\ifwasdraft
\wasdraftfalse

%  \GRAPHIC{ body }                                  %#1
%          { draft name }                            %#2
%          { contentswidth (scalar)  }               %#3
%          { contentsheight (scalar) }               %#4
%          { vertical shift when in-line (scalar) }  %#5
\def\GRAPHIC#1#2#3#4#5{%
 \ifnum\draft=\@ne\draftbox{#2}{#3}{#4}{#5}%
  \else\graffile{#1}{#3}{#4}{#5}%
  \fi
 }%
%
\def\addtoLaTeXparams#1{%
    \edef\LaTeXparams{\LaTeXparams #1}}%
%
% JCS -  added a switch BoxFrame that can 
% be set by including X in the frame params.
% If set a box is drawn around the frame.

\newif\ifBoxFrame \BoxFramefalse
\newif\ifOverFrame \OverFramefalse
\newif\ifUnderFrame \UnderFramefalse

\def\BOXTHEFRAME#1{%
   \hbox{%
      \ifBoxFrame
         \frame{#1}%
      \else
         {#1}%
      \fi
   }%
}


\def\doFRAMEparams#1{\BoxFramefalse\OverFramefalse\UnderFramefalse\readFRAMEparams#1\end}%
\def\readFRAMEparams#1{%
 \ifx#1\end%
  \let\next=\relax
  \else
  \ifx#1i\dispkind=\z@\fi
  \ifx#1d\dispkind=\@ne\fi
  \ifx#1f\dispkind=\tw@\fi
  \ifx#1t\addtoLaTeXparams{t}\fi
  \ifx#1b\addtoLaTeXparams{b}\fi
  \ifx#1p\addtoLaTeXparams{p}\fi
  \ifx#1h\addtoLaTeXparams{h}\fi
  \ifx#1X\BoxFrametrue\fi
  \ifx#1O\OverFrametrue\fi
  \ifx#1U\UnderFrametrue\fi
  \ifx#1w
    \ifnum\draft=1\wasdrafttrue\else\wasdraftfalse\fi
    \draft=\@ne
  \fi
  \let\next=\readFRAMEparams
  \fi
 \next
 }%
%
%Macro for In-line graphics object
%   \IFRAME{ contentswidth (scalar)  }               %#1
%          { contentsheight (scalar) }               %#2
%          { vertical shift when in-line (scalar) }  %#3
%          { draft name }                            %#4
%          { body }                                  %#5
%          { caption}                                %#6


\def\IFRAME#1#2#3#4#5#6{%
      \bgroup
      \let\QCTOptA\empty
      \let\QCTOptB\empty
      \let\QCBOptA\empty
      \let\QCBOptB\empty
      #6%
      \parindent=0pt%
      \leftskip=0pt
      \rightskip=0pt
      \setbox0 = \hbox{\QCBOptA}%
      \@tempdima = #1\relax
      \ifOverFrame
          % Do this later
          \typeout{This is not implemented yet}%
          \show\HELP
      \else
         \ifdim\wd0>\@tempdima
            \advance\@tempdima by \@tempdima
            \ifdim\wd0 >\@tempdima
               \textwidth=\@tempdima
               \setbox1 =\vbox{%
                  \noindent\hbox to \@tempdima{\hfill\GRAPHIC{#5}{#4}{#1}{#2}{#3}\hfill}\\%
                  \noindent\hbox to \@tempdima{\parbox[b]{\@tempdima}{\QCBOptA}}%
               }%
               \wd1=\@tempdima
            \else
               \textwidth=\wd0
               \setbox1 =\vbox{%
                 \noindent\hbox to \wd0{\hfill\GRAPHIC{#5}{#4}{#1}{#2}{#3}\hfill}\\%
                 \noindent\hbox{\QCBOptA}%
               }%
               \wd1=\wd0
            \fi
         \else
            %\show\BBB
            \ifdim\wd0>0pt
              \hsize=\@tempdima
              \setbox1 =\vbox{%
                \unskip\GRAPHIC{#5}{#4}{#1}{#2}{0pt}%
                \break
                \unskip\hbox to \@tempdima{\hfill \QCBOptA\hfill}%
              }%
              \wd1=\@tempdima
           \else
              \hsize=\@tempdima
              \setbox1 =\vbox{%
                \unskip\GRAPHIC{#5}{#4}{#1}{#2}{0pt}%
              }%
              \wd1=\@tempdima
           \fi
         \fi
         \@tempdimb=\ht1
         \advance\@tempdimb by \dp1
         \advance\@tempdimb by -#2%
         \advance\@tempdimb by #3%
         \leavevmode
         \raise -\@tempdimb \hbox{\box1}%
      \fi
      \egroup%
}%
%
%Macro for Display graphics object
%   \DFRAME{ contentswidth (scalar)  }               %#1
%          { contentsheight (scalar) }               %#2
%          { draft label }                           %#3
%          { name }                                  %#4
%          { caption}                                %#5
\def\DFRAME#1#2#3#4#5{%
 \begin{center}
     \let\QCTOptA\empty
     \let\QCTOptB\empty
     \let\QCBOptA\empty
     \let\QCBOptB\empty
     \ifOverFrame 
        #5\QCTOptA\par
     \fi
     \GRAPHIC{#4}{#3}{#1}{#2}{\z@}
     \ifUnderFrame 
        \nobreak\par #5\QCBOptA
     \fi
 \end{center}%
 }%
%
%Macro for Floating graphic object
%   \FFRAME{ framedata f|i tbph x F|T }              %#1
%          { contentswidth (scalar)  }               %#2
%          { contentsheight (scalar) }               %#3
%          { caption }                               %#4
%          { label }                                 %#5
%          { draft name }                            %#6
%          { body }                                  %#7
\def\FFRAME#1#2#3#4#5#6#7{%
 \begin{figure}[#1]%
  \let\QCTOptA\empty
  \let\QCTOptB\empty
  \let\QCBOptA\empty
  \let\QCBOptB\empty
  \ifOverFrame
    #4
    \ifx\QCTOptA\empty
    \else
      \ifx\QCTOptB\empty
        \caption{\QCTOptA}%
      \else
        \caption[\QCTOptB]{\QCTOptA}%
      \fi
    \fi
    \ifUnderFrame\else
      \label{#5}%
    \fi
  \else
    \UnderFrametrue%
  \fi
  \begin{center}\GRAPHIC{#7}{#6}{#2}{#3}{\z@}\end{center}%
  \ifUnderFrame
    #4
    \ifx\QCBOptA\empty
      \caption{}%
    \else
      \ifx\QCBOptB\empty
        \caption{\QCBOptA}%
      \else
        \caption[\QCBOptB]{\QCBOptA}%
      \fi
    \fi
    \label{#5}%
  \fi
  \end{figure}%
 }%
%
%
%    \FRAME{ framedata f|i tbph x F|T }              %#1
%          { contentswidth (scalar)  }               %#2
%          { contentsheight (scalar) }               %#3
%          { vertical shift when in-line (scalar) }  %#4
%          { caption }                               %#5
%          { label }                                 %#6
%          { name }                                  %#7
%          { body }                                  %#8
%
%    framedata is a string which can contain the following
%    characters: idftbphxFT
%    Their meaning is as follows:
%             i, d or f : in-line, display, or floating
%             t,b,p,h   : LaTeX floating placement options
%             x         : fit contents box to contents
%             F or T    : Figure or Table. 
%                         Later this can expand
%                         to a more general float class.
%
%
\newcount\dispkind%

\def\makeactives{
  \catcode`\"=\active
  \catcode`\;=\active
  \catcode`\:=\active
  \catcode`\'=\active
  \catcode`\~=\active
}
\bgroup
   \makeactives
   \gdef\activesoff{%
      \def"{\string"}
      \def;{\string;}
      \def:{\string:}
      \def'{\string'}
      \def~{\string~}
      %\bbl@deactivate{"}%
      %\bbl@deactivate{;}%
      %\bbl@deactivate{:}%
      %\bbl@deactivate{'}%
    }
\egroup

\def\FRAME#1#2#3#4#5#6#7#8{%
 \bgroup
 \@ifundefined{bbl@deactivate}{}{\activesoff}
 \ifnum\draft=\@ne
   \wasdrafttrue
 \else
   \wasdraftfalse%
 \fi
 \def\LaTeXparams{}%
 \dispkind=\z@
 \def\LaTeXparams{}%
 \doFRAMEparams{#1}%
 \ifnum\dispkind=\z@\IFRAME{#2}{#3}{#4}{#7}{#8}{#5}\else
  \ifnum\dispkind=\@ne\DFRAME{#2}{#3}{#7}{#8}{#5}\else
   \ifnum\dispkind=\tw@
    \edef\@tempa{\noexpand\FFRAME{\LaTeXparams}}%
    \@tempa{#2}{#3}{#5}{#6}{#7}{#8}%
    \fi
   \fi
  \fi
  \ifwasdraft\draft=1\else\draft=0\fi{}%
  \egroup
 }%
%
% This macro added to let SW gobble a parameter that
% should not be passed on and expanded. 

\def\TEXUX#1{"texux"}

%
% Macros for text attributes:
%
\def\BF#1{{\bf {#1}}}%
\def\NEG#1{\leavevmode\hbox{\rlap{\thinspace/}{$#1$}}}%
%
%%%%%%%%%%%%%%%%%%%%%%%%%%%%%%%%%%%%%%%%%%%%%%%%%%%%%%%%%%%%%%%%%%%%%%%%
%
%
% macros for user - defined functions
\def\func#1{\mathop{\rm #1}}%
\def\limfunc#1{\mathop{\rm #1}}%

%
% miscellaneous 
%\long\def\QQQ#1#2{}%
\long\def\QQQ#1#2{%
     \long\expandafter\def\csname#1\endcsname{#2}}%
%\def\QTP#1{}% JCS - this was changed becuase style editor will define QTP
\@ifundefined{QTP}{\def\QTP#1{}}{}
\@ifundefined{QEXCLUDE}{\def\QEXCLUDE#1{}}{}
%\@ifundefined{Qcb}{\def\Qcb#1{#1}}{}
%\@ifundefined{Qct}{\def\Qct#1{#1}}{}
\@ifundefined{Qlb}{\def\Qlb#1{#1}}{}
\@ifundefined{Qlt}{\def\Qlt#1{#1}}{}
\def\QWE{}%
\long\def\QQA#1#2{}%
%\def\QTR#1#2{{\em #2}}% Always \em!!!
%\def\QTR#1#2{\mbox{\begin{#1}#2\end{#1}}}%cb%%%
\def\QTR#1#2{{\csname#1\endcsname #2}}%(gp) Is this the best?
\long\def\TeXButton#1#2{#2}%
\long\def\QSubDoc#1#2{#2}%
\def\EXPAND#1[#2]#3{}%
\def\NOEXPAND#1[#2]#3{}%
\def\PROTECTED{}%
\def\LaTeXparent#1{}%
\def\ChildStyles#1{}%
\def\ChildDefaults#1{}%
\def\QTagDef#1#2#3{}%
%
% Macros for style editor docs
\@ifundefined{StyleEditBeginDoc}{\def\StyleEditBeginDoc{\relax}}{}
%
% Macros for footnotes
\def\QQfnmark#1{\footnotemark}
\def\QQfntext#1#2{\addtocounter{footnote}{#1}\footnotetext{#2}}
%
% Macros for indexing.
\def\MAKEINDEX{\makeatletter\input gnuindex.sty\makeatother\makeindex}%	
\@ifundefined{INDEX}{\def\INDEX#1#2{}{}}{}%
\@ifundefined{SUBINDEX}{\def\SUBINDEX#1#2#3{}{}{}}{}%
\@ifundefined{initial}%  
   {\def\initial#1{\bigbreak{\raggedright\large\bf #1}\kern 2\p@\penalty3000}}%
   {}%
\@ifundefined{entry}{\def\entry#1#2{\item {#1}, #2}}{}%
\@ifundefined{primary}{\def\primary#1{\item {#1}}}{}%
\@ifundefined{secondary}{\def\secondary#1#2{\subitem {#1}, #2}}{}%
%
%
\@ifundefined{ZZZ}{}{\MAKEINDEX\makeatletter}%
%
% Attempts to avoid problems with other styles
\@ifundefined{abstract}{%
 \def\abstract{%
  \if@twocolumn
   \section*{Abstract (Not appropriate in this style!)}%
   \else \small 
   \begin{center}{\bf Abstract\vspace{-.5em}\vspace{\z@}}\end{center}%
   \quotation 
   \fi
  }%
 }{%
 }%
\@ifundefined{endabstract}{\def\endabstract
  {\if@twocolumn\else\endquotation\fi}}{}%
\@ifundefined{maketitle}{\def\maketitle#1{}}{}%
\@ifundefined{affiliation}{\def\affiliation#1{}}{}%
\@ifundefined{proof}{\def\proof{\noindent{\bfseries Proof. }}}{}%
\@ifundefined{endproof}{\def\endproof{\mbox{\ \rule{.1in}{.1in}}}}{}%
\@ifundefined{newfield}{\def\newfield#1#2{}}{}%
\@ifundefined{chapter}{\def\chapter#1{\par(Chapter head:)#1\par }%
 \newcount\c@chapter}{}%
\@ifundefined{part}{\def\part#1{\par(Part head:)#1\par }}{}%
\@ifundefined{section}{\def\section#1{\par(Section head:)#1\par }}{}%
\@ifundefined{subsection}{\def\subsection#1%
 {\par(Subsection head:)#1\par }}{}%
\@ifundefined{subsubsection}{\def\subsubsection#1%
 {\par(Subsubsection head:)#1\par }}{}%
\@ifundefined{paragraph}{\def\paragraph#1%
 {\par(Subsubsubsection head:)#1\par }}{}%
\@ifundefined{subparagraph}{\def\subparagraph#1%
 {\par(Subsubsubsubsection head:)#1\par }}{}%
%%%%%%%%%%%%%%%%%%%%%%%%%%%%%%%%%%%%%%%%%%%%%%%%%%%%%%%%%%%%%%%%%%%%%%%%
% These symbols are not recognized by LaTeX
\@ifundefined{therefore}{\def\therefore{}}{}%
\@ifundefined{backepsilon}{\def\backepsilon{}}{}%
\@ifundefined{yen}{\def\yen{\hbox{\rm\rlap=Y}}}{}%
\@ifundefined{registered}{%
   \def\registered{\relax\ifmmode{}\r@gistered
                    \else$\m@th\r@gistered$\fi}%
 \def\r@gistered{^{\ooalign
  {\hfil\raise.07ex\hbox{$\scriptstyle\rm\text{R}$}\hfil\crcr
  \mathhexbox20D}}}}{}%
\@ifundefined{Eth}{\def\Eth{}}{}%
\@ifundefined{eth}{\def\eth{}}{}%
\@ifundefined{Thorn}{\def\Thorn{}}{}%
\@ifundefined{thorn}{\def\thorn{}}{}%
% A macro to allow any symbol that requires math to appear in text
\def\TEXTsymbol#1{\mbox{$#1$}}%
\@ifundefined{degree}{\def\degree{{}^{\circ}}}{}%
%
% macros for T3TeX files
\newdimen\theight
\def\Column{%
 \vadjust{\setbox\z@=\hbox{\scriptsize\quad\quad tcol}%
  \theight=\ht\z@\advance\theight by \dp\z@\advance\theight by \lineskip
  \kern -\theight \vbox to \theight{%
   \rightline{\rlap{\box\z@}}%
   \vss
   }%
  }%
 }%
%
\def\qed{%
 \ifhmode\unskip\nobreak\fi\ifmmode\ifinner\else\hskip5\p@\fi\fi
 \hbox{\hskip5\p@\vrule width4\p@ height6\p@ depth1.5\p@\hskip\p@}%
 }%
%
\def\cents{\hbox{\rm\rlap/c}}%
\def\miss{\hbox{\vrule height2\p@ width 2\p@ depth\z@}}%
%\def\miss{\hbox{.}}%        %another possibility 
%
\def\vvert{\Vert}%           %always translated to \left| or \right|
%
\def\tcol#1{{\baselineskip=6\p@ \vcenter{#1}} \Column}  %
%
\def\dB{\hbox{{}}}%                 %dummy entry in column 
\def\mB#1{\hbox{$#1$}}%             %column entry
\def\nB#1{\hbox{#1}}%               %column entry (not math)
%
%\newcount\notenumber
%\def\clearnotenumber{\notenumber=0}
%\def\note{\global\advance\notenumber by 1
% \footnote{$^{\the\notenumber}$}}
%\def\note{\global\advance\notenumber by 1
\def\note{$^{\dag}}%
%
%

\def\newfmtname{LaTeX2e}
\def\chkcompat{%
   \if@compatibility
   \else
     \usepackage{latexsym}
   \fi
}

\ifx\fmtname\newfmtname
  \DeclareOldFontCommand{\rm}{\normalfont\rmfamily}{\mathrm}
  \DeclareOldFontCommand{\sf}{\normalfont\sffamily}{\mathsf}
  \DeclareOldFontCommand{\tt}{\normalfont\ttfamily}{\mathtt}
  \DeclareOldFontCommand{\bf}{\normalfont\bfseries}{\mathbf}
  \DeclareOldFontCommand{\it}{\normalfont\itshape}{\mathit}
  \DeclareOldFontCommand{\sl}{\normalfont\slshape}{\@nomath\sl}
  \DeclareOldFontCommand{\sc}{\normalfont\scshape}{\@nomath\sc}
  \chkcompat
\fi

%
% Greek bold macros
% Redefine all of the math symbols 
% which might be bolded	 - there are 
% probably others to add to this list

\def\alpha{\Greekmath 010B }%
\def\beta{\Greekmath 010C }%
\def\gamma{\Greekmath 010D }%
\def\delta{\Greekmath 010E }%
\def\epsilon{\Greekmath 010F }%
\def\zeta{\Greekmath 0110 }%
\def\eta{\Greekmath 0111 }%
\def\theta{\Greekmath 0112 }%
\def\iota{\Greekmath 0113 }%
\def\kappa{\Greekmath 0114 }%
\def\lambda{\Greekmath 0115 }%
\def\mu{\Greekmath 0116 }%
\def\nu{\Greekmath 0117 }%
\def\xi{\Greekmath 0118 }%
\def\pi{\Greekmath 0119 }%
\def\rho{\Greekmath 011A }%
\def\sigma{\Greekmath 011B }%
\def\tau{\Greekmath 011C }%
\def\upsilon{\Greekmath 011D }%
\def\phi{\Greekmath 011E }%
\def\chi{\Greekmath 011F }%
\def\psi{\Greekmath 0120 }%
\def\omega{\Greekmath 0121 }%
\def\varepsilon{\Greekmath 0122 }%
\def\vartheta{\Greekmath 0123 }%
\def\varpi{\Greekmath 0124 }%
\def\varrho{\Greekmath 0125 }%
\def\varsigma{\Greekmath 0126 }%
\def\varphi{\Greekmath 0127 }%

\def\nabla{\Greekmath 0272 }
\def\FindBoldGroup{%
   {\setbox0=\hbox{$\mathbf{x\global\edef\theboldgroup{\the\mathgroup}}$}}%
}

\def\Greekmath#1#2#3#4{%
    \if@compatibility
        \ifnum\mathgroup=\symbold
           \mathchoice{\mbox{\boldmath$\displaystyle\mathchar"#1#2#3#4$}}%
                      {\mbox{\boldmath$\textstyle\mathchar"#1#2#3#4$}}%
                      {\mbox{\boldmath$\scriptstyle\mathchar"#1#2#3#4$}}%
                      {\mbox{\boldmath$\scriptscriptstyle\mathchar"#1#2#3#4$}}%
        \else
           \mathchar"#1#2#3#4% 
        \fi 
    \else 
        \FindBoldGroup
        \ifnum\mathgroup=\theboldgroup % For 2e
           \mathchoice{\mbox{\boldmath$\displaystyle\mathchar"#1#2#3#4$}}%
                      {\mbox{\boldmath$\textstyle\mathchar"#1#2#3#4$}}%
                      {\mbox{\boldmath$\scriptstyle\mathchar"#1#2#3#4$}}%
                      {\mbox{\boldmath$\scriptscriptstyle\mathchar"#1#2#3#4$}}%
        \else
           \mathchar"#1#2#3#4% 
        \fi     	    
	  \fi}

\newif\ifGreekBold  \GreekBoldfalse
\let\SAVEPBF=\pbf
\def\pbf{\GreekBoldtrue\SAVEPBF}%
%

\@ifundefined{theorem}{\newtheorem{theorem}{Theorem}}{}
\@ifundefined{lemma}{\newtheorem{lemma}[theorem]{Lemma}}{}
\@ifundefined{corollary}{\newtheorem{corollary}[theorem]{Corollary}}{}
\@ifundefined{conjecture}{\newtheorem{conjecture}[theorem]{Conjecture}}{}
\@ifundefined{proposition}{\newtheorem{proposition}[theorem]{Proposition}}{}
\@ifundefined{axiom}{\newtheorem{axiom}{Axiom}}{}
\@ifundefined{remark}{\newtheorem{remark}{Remark}}{}
\@ifundefined{example}{\newtheorem{example}{Example}}{}
\@ifundefined{exercise}{\newtheorem{exercise}{Exercise}}{}
\@ifundefined{definition}{\newtheorem{definition}{Definition}}{}


\@ifundefined{mathletters}{%
  %\def\theequation{\arabic{equation}}
  \newcounter{equationnumber}  
  \def\mathletters{%
     \addtocounter{equation}{1}
     \edef\@currentlabel{\theequation}%
     \setcounter{equationnumber}{\c@equation}
     \setcounter{equation}{0}%
     \edef\theequation{\@currentlabel\noexpand\alph{equation}}%
  }
  \def\endmathletters{%
     \setcounter{equation}{\value{equationnumber}}%
  }
}{}

%Logos
\@ifundefined{BibTeX}{%
    \def\BibTeX{{\rm B\kern-.05em{\sc i\kern-.025em b}\kern-.08em
                 T\kern-.1667em\lower.7ex\hbox{E}\kern-.125emX}}}{}%
\@ifundefined{AmS}%
    {\def\AmS{{\protect\usefont{OMS}{cmsy}{m}{n}%
                A\kern-.1667em\lower.5ex\hbox{M}\kern-.125emS}}}{}%
\@ifundefined{AmSTeX}{\def\AmSTeX{\protect\AmS-\protect\TeX\@}}{}%
%

%%%%%%%%%%%%%%%%%%%%%%%%%%%%%%%%%%%%%%%%%%%%%%%%%%%%%%%%%%%%%%%%%%%%%%%
% NOTE: The rest of this file is read only if amstex has not been
% loaded.  This section is used to define amstex constructs in the
% event they have not been defined.
%
%
\ifx\ds@amstex\relax
   \message{amstex already loaded}\makeatother\endinput% 2.09 compatability
\else
   \@ifpackageloaded{amstex}%
      {\message{amstex already loaded}\makeatother\endinput}
      {}
   \@ifpackageloaded{amsgen}%
      {\message{amsgen already loaded}\makeatother\endinput}
      {}
\fi
%%%%%%%%%%%%%%%%%%%%%%%%%%%%%%%%%%%%%%%%%%%%%%%%%%%%%%%%%%%%%%%%%%%%%%%%
%%
%
%
%  Macros to define some AMS LaTeX constructs when 
%  AMS LaTeX has not been loaded
% 
% These macros are copied from the AMS-TeX package for doing
% multiple integrals.
%
\let\DOTSI\relax
\def\RIfM@{\relax\ifmmode}%
\def\FN@{\futurelet\next}%
\newcount\intno@
\def\iint{\DOTSI\intno@\tw@\FN@\ints@}%
\def\iiint{\DOTSI\intno@\thr@@\FN@\ints@}%
\def\iiiint{\DOTSI\intno@4 \FN@\ints@}%
\def\idotsint{\DOTSI\intno@\z@\FN@\ints@}%
\def\ints@{\findlimits@\ints@@}%
\newif\iflimtoken@
\newif\iflimits@
\def\findlimits@{\limtoken@true\ifx\next\limits\limits@true
 \else\ifx\next\nolimits\limits@false\else
 \limtoken@false\ifx\ilimits@\nolimits\limits@false\else
 \ifinner\limits@false\else\limits@true\fi\fi\fi\fi}%
\def\multint@{\int\ifnum\intno@=\z@\intdots@                          %1
 \else\intkern@\fi                                                    %2
 \ifnum\intno@>\tw@\int\intkern@\fi                                   %3
 \ifnum\intno@>\thr@@\int\intkern@\fi                                 %4
 \int}%                                                               %5
\def\multintlimits@{\intop\ifnum\intno@=\z@\intdots@\else\intkern@\fi
 \ifnum\intno@>\tw@\intop\intkern@\fi
 \ifnum\intno@>\thr@@\intop\intkern@\fi\intop}%
\def\intic@{%
    \mathchoice{\hskip.5em}{\hskip.4em}{\hskip.4em}{\hskip.4em}}%
\def\negintic@{\mathchoice
 {\hskip-.5em}{\hskip-.4em}{\hskip-.4em}{\hskip-.4em}}%
\def\ints@@{\iflimtoken@                                              %1
 \def\ints@@@{\iflimits@\negintic@
   \mathop{\intic@\multintlimits@}\limits                             %2
  \else\multint@\nolimits\fi                                          %3
  \eat@}%                                                             %4
 \else                                                                %5
 \def\ints@@@{\iflimits@\negintic@
  \mathop{\intic@\multintlimits@}\limits\else
  \multint@\nolimits\fi}\fi\ints@@@}%
\def\intkern@{\mathchoice{\!\!\!}{\!\!}{\!\!}{\!\!}}%
\def\plaincdots@{\mathinner{\cdotp\cdotp\cdotp}}%
\def\intdots@{\mathchoice{\plaincdots@}%
 {{\cdotp}\mkern1.5mu{\cdotp}\mkern1.5mu{\cdotp}}%
 {{\cdotp}\mkern1mu{\cdotp}\mkern1mu{\cdotp}}%
 {{\cdotp}\mkern1mu{\cdotp}\mkern1mu{\cdotp}}}%
%
%
%  These macros are for doing the AMS \text{} construct
%
\def\RIfM@{\relax\protect\ifmmode}
\def\text{\RIfM@\expandafter\text@\else\expandafter\mbox\fi}
\let\nfss@text\text
\def\text@#1{\mathchoice
   {\textdef@\displaystyle\f@size{#1}}%
   {\textdef@\textstyle\tf@size{\firstchoice@false #1}}%
   {\textdef@\textstyle\sf@size{\firstchoice@false #1}}%
   {\textdef@\textstyle \ssf@size{\firstchoice@false #1}}%
   \glb@settings}

\def\textdef@#1#2#3{\hbox{{%
                    \everymath{#1}%
                    \let\f@size#2\selectfont
                    #3}}}
\newif\iffirstchoice@
\firstchoice@true
%
%    Old Scheme for \text
%
%\def\rmfam{\z@}%
%\newif\iffirstchoice@
%\firstchoice@true
%\def\textfonti{\the\textfont\@ne}%
%\def\textfontii{\the\textfont\tw@}%
%\def\text{\RIfM@\expandafter\text@\else\expandafter\text@@\fi}%
%\def\text@@#1{\leavevmode\hbox{#1}}%
%\def\text@#1{\mathchoice
% {\hbox{\everymath{\displaystyle}\def\textfonti{\the\textfont\@ne}%
%  \def\textfontii{\the\textfont\tw@}\textdef@@ T#1}}%
% {\hbox{\firstchoice@false
%  \everymath{\textstyle}\def\textfonti{\the\textfont\@ne}%
%  \def\textfontii{\the\textfont\tw@}\textdef@@ T#1}}%
% {\hbox{\firstchoice@false
%  \everymath{\scriptstyle}\def\textfonti{\the\scriptfont\@ne}%
%  \def\textfontii{\the\scriptfont\tw@}\textdef@@ S\rm#1}}%
% {\hbox{\firstchoice@false
%  \everymath{\scriptscriptstyle}\def\textfonti
%  {\the\scriptscriptfont\@ne}%
%  \def\textfontii{\the\scriptscriptfont\tw@}\textdef@@ s\rm#1}}}%
%\def\textdef@@#1{\textdef@#1\rm\textdef@#1\bf\textdef@#1\sl
%    \textdef@#1\it}%
%\def\DN@{\def\next@}%
%\def\eat@#1{}%
%\def\textdef@#1#2{%
% \DN@{\csname\expandafter\eat@\string#2fam\endcsname}%
% \if S#1\edef#2{\the\scriptfont\next@\relax}%
% \else\if s#1\edef#2{\the\scriptscriptfont\next@\relax}%
% \else\edef#2{\the\textfont\next@\relax}\fi\fi}%
%
%
%These are the AMS constructs for multiline limits.
%
\def\Let@{\relax\iffalse{\fi\let\\=\cr\iffalse}\fi}%
\def\vspace@{\def\vspace##1{\crcr\noalign{\vskip##1\relax}}}%
\def\multilimits@{\bgroup\vspace@\Let@
 \baselineskip\fontdimen10 \scriptfont\tw@
 \advance\baselineskip\fontdimen12 \scriptfont\tw@
 \lineskip\thr@@\fontdimen8 \scriptfont\thr@@
 \lineskiplimit\lineskip
 \vbox\bgroup\ialign\bgroup\hfil$\m@th\scriptstyle{##}$\hfil\crcr}%
\def\Sb{_\multilimits@}%
\def\endSb{\crcr\egroup\egroup\egroup}%
\def\Sp{^\multilimits@}%
\let\endSp\endSb
%
%
%These are AMS constructs for horizontal arrows
%
\newdimen\ex@
\ex@.2326ex
\def\rightarrowfill@#1{$#1\m@th\mathord-\mkern-6mu\cleaders
 \hbox{$#1\mkern-2mu\mathord-\mkern-2mu$}\hfill
 \mkern-6mu\mathord\rightarrow$}%
\def\leftarrowfill@#1{$#1\m@th\mathord\leftarrow\mkern-6mu\cleaders
 \hbox{$#1\mkern-2mu\mathord-\mkern-2mu$}\hfill\mkern-6mu\mathord-$}%
\def\leftrightarrowfill@#1{$#1\m@th\mathord\leftarrow
\mkern-6mu\cleaders
 \hbox{$#1\mkern-2mu\mathord-\mkern-2mu$}\hfill
 \mkern-6mu\mathord\rightarrow$}%
\def\overrightarrow{\mathpalette\overrightarrow@}%
\def\overrightarrow@#1#2{\vbox{\ialign{##\crcr\rightarrowfill@#1\crcr
 \noalign{\kern-\ex@\nointerlineskip}$\m@th\hfil#1#2\hfil$\crcr}}}%
\let\overarrow\overrightarrow
\def\overleftarrow{\mathpalette\overleftarrow@}%
\def\overleftarrow@#1#2{\vbox{\ialign{##\crcr\leftarrowfill@#1\crcr
 \noalign{\kern-\ex@\nointerlineskip}$\m@th\hfil#1#2\hfil$\crcr}}}%
\def\overleftrightarrow{\mathpalette\overleftrightarrow@}%
\def\overleftrightarrow@#1#2{\vbox{\ialign{##\crcr
   \leftrightarrowfill@#1\crcr
 \noalign{\kern-\ex@\nointerlineskip}$\m@th\hfil#1#2\hfil$\crcr}}}%
\def\underrightarrow{\mathpalette\underrightarrow@}%
\def\underrightarrow@#1#2{\vtop{\ialign{##\crcr$\m@th\hfil#1#2\hfil
  $\crcr\noalign{\nointerlineskip}\rightarrowfill@#1\crcr}}}%
\let\underarrow\underrightarrow
\def\underleftarrow{\mathpalette\underleftarrow@}%
\def\underleftarrow@#1#2{\vtop{\ialign{##\crcr$\m@th\hfil#1#2\hfil
  $\crcr\noalign{\nointerlineskip}\leftarrowfill@#1\crcr}}}%
\def\underleftrightarrow{\mathpalette\underleftrightarrow@}%
\def\underleftrightarrow@#1#2{\vtop{\ialign{##\crcr$\m@th
  \hfil#1#2\hfil$\crcr
 \noalign{\nointerlineskip}\leftrightarrowfill@#1\crcr}}}%
%%%%%%%%%%%%%%%%%%%%%

% 94.0815 by Jon:

\def\qopnamewl@#1{\mathop{\operator@font#1}\nlimits@}
\let\nlimits@\displaylimits
\def\setboxz@h{\setbox\z@\hbox}


\def\varlim@#1#2{\mathop{\vtop{\ialign{##\crcr
 \hfil$#1\m@th\operator@font lim$\hfil\crcr
 \noalign{\nointerlineskip}#2#1\crcr
 \noalign{\nointerlineskip\kern-\ex@}\crcr}}}}

 \def\rightarrowfill@#1{\m@th\setboxz@h{$#1-$}\ht\z@\z@
  $#1\copy\z@\mkern-6mu\cleaders
  \hbox{$#1\mkern-2mu\box\z@\mkern-2mu$}\hfill
  \mkern-6mu\mathord\rightarrow$}
\def\leftarrowfill@#1{\m@th\setboxz@h{$#1-$}\ht\z@\z@
  $#1\mathord\leftarrow\mkern-6mu\cleaders
  \hbox{$#1\mkern-2mu\copy\z@\mkern-2mu$}\hfill
  \mkern-6mu\box\z@$}


\def\projlim{\qopnamewl@{proj\,lim}}
\def\injlim{\qopnamewl@{inj\,lim}}
\def\varinjlim{\mathpalette\varlim@\rightarrowfill@}
\def\varprojlim{\mathpalette\varlim@\leftarrowfill@}
\def\varliminf{\mathpalette\varliminf@{}}
\def\varliminf@#1{\mathop{\underline{\vrule\@depth.2\ex@\@width\z@
   \hbox{$#1\m@th\operator@font lim$}}}}
\def\varlimsup{\mathpalette\varlimsup@{}}
\def\varlimsup@#1{\mathop{\overline
  {\hbox{$#1\m@th\operator@font lim$}}}}

%
%%%%%%%%%%%%%%%%%%%%%%%%%%%%%%%%%%%%%%%%%%%%%%%%%%%%%%%%%%%%%%%%%%%%%
%
\def\tfrac#1#2{{\textstyle {#1 \over #2}}}%
\def\dfrac#1#2{{\displaystyle {#1 \over #2}}}%
\def\binom#1#2{{#1 \choose #2}}%
\def\tbinom#1#2{{\textstyle {#1 \choose #2}}}%
\def\dbinom#1#2{{\displaystyle {#1 \choose #2}}}%
\def\QATOP#1#2{{#1 \atop #2}}%
\def\QTATOP#1#2{{\textstyle {#1 \atop #2}}}%
\def\QDATOP#1#2{{\displaystyle {#1 \atop #2}}}%
\def\QABOVE#1#2#3{{#2 \above#1 #3}}%
\def\QTABOVE#1#2#3{{\textstyle {#2 \above#1 #3}}}%
\def\QDABOVE#1#2#3{{\displaystyle {#2 \above#1 #3}}}%
\def\QOVERD#1#2#3#4{{#3 \overwithdelims#1#2 #4}}%
\def\QTOVERD#1#2#3#4{{\textstyle {#3 \overwithdelims#1#2 #4}}}%
\def\QDOVERD#1#2#3#4{{\displaystyle {#3 \overwithdelims#1#2 #4}}}%
\def\QATOPD#1#2#3#4{{#3 \atopwithdelims#1#2 #4}}%
\def\QTATOPD#1#2#3#4{{\textstyle {#3 \atopwithdelims#1#2 #4}}}%
\def\QDATOPD#1#2#3#4{{\displaystyle {#3 \atopwithdelims#1#2 #4}}}%
\def\QABOVED#1#2#3#4#5{{#4 \abovewithdelims#1#2#3 #5}}%
\def\QTABOVED#1#2#3#4#5{{\textstyle 
   {#4 \abovewithdelims#1#2#3 #5}}}%
\def\QDABOVED#1#2#3#4#5{{\displaystyle 
   {#4 \abovewithdelims#1#2#3 #5}}}%
%
% Macros for text size operators:

%JCS - added braces and \mathop around \displaystyle\int, etc.
%
\def\tint{\mathop{\textstyle \int}}%
\def\tiint{\mathop{\textstyle \iint }}%
\def\tiiint{\mathop{\textstyle \iiint }}%
\def\tiiiint{\mathop{\textstyle \iiiint }}%
\def\tidotsint{\mathop{\textstyle \idotsint }}%
\def\toint{\mathop{\textstyle \oint}}%
\def\tsum{\mathop{\textstyle \sum }}%
\def\tprod{\mathop{\textstyle \prod }}%
\def\tbigcap{\mathop{\textstyle \bigcap }}%
\def\tbigwedge{\mathop{\textstyle \bigwedge }}%
\def\tbigoplus{\mathop{\textstyle \bigoplus }}%
\def\tbigodot{\mathop{\textstyle \bigodot }}%
\def\tbigsqcup{\mathop{\textstyle \bigsqcup }}%
\def\tcoprod{\mathop{\textstyle \coprod }}%
\def\tbigcup{\mathop{\textstyle \bigcup }}%
\def\tbigvee{\mathop{\textstyle \bigvee }}%
\def\tbigotimes{\mathop{\textstyle \bigotimes }}%
\def\tbiguplus{\mathop{\textstyle \biguplus }}%
%
%
%Macros for display size operators:
%

\def\dint{\mathop{\displaystyle \int}}%
\def\diint{\mathop{\displaystyle \iint }}%
\def\diiint{\mathop{\displaystyle \iiint }}%
\def\diiiint{\mathop{\displaystyle \iiiint }}%
\def\didotsint{\mathop{\displaystyle \idotsint }}%
\def\doint{\mathop{\displaystyle \oint}}%
\def\dsum{\mathop{\displaystyle \sum }}%
\def\dprod{\mathop{\displaystyle \prod }}%
\def\dbigcap{\mathop{\displaystyle \bigcap }}%
\def\dbigwedge{\mathop{\displaystyle \bigwedge }}%
\def\dbigoplus{\mathop{\displaystyle \bigoplus }}%
\def\dbigodot{\mathop{\displaystyle \bigodot }}%
\def\dbigsqcup{\mathop{\displaystyle \bigsqcup }}%
\def\dcoprod{\mathop{\displaystyle \coprod }}%
\def\dbigcup{\mathop{\displaystyle \bigcup }}%
\def\dbigvee{\mathop{\displaystyle \bigvee }}%
\def\dbigotimes{\mathop{\displaystyle \bigotimes }}%
\def\dbiguplus{\mathop{\displaystyle \biguplus }}%
%
%Companion to stackrel
\def\stackunder#1#2{\mathrel{\mathop{#2}\limits_{#1}}}%
%
%
% These are AMS environments that will be defined to
% be verbatims if amstex has not actually been 
% loaded
%
%
\begingroup \catcode `|=0 \catcode `[= 1
\catcode`]=2 \catcode `\{=12 \catcode `\}=12
\catcode`\\=12 
|gdef|@alignverbatim#1\end{align}[#1|end[align]]
|gdef|@salignverbatim#1\end{align*}[#1|end[align*]]

|gdef|@alignatverbatim#1\end{alignat}[#1|end[alignat]]
|gdef|@salignatverbatim#1\end{alignat*}[#1|end[alignat*]]

|gdef|@xalignatverbatim#1\end{xalignat}[#1|end[xalignat]]
|gdef|@sxalignatverbatim#1\end{xalignat*}[#1|end[xalignat*]]

|gdef|@gatherverbatim#1\end{gather}[#1|end[gather]]
|gdef|@sgatherverbatim#1\end{gather*}[#1|end[gather*]]

|gdef|@gatherverbatim#1\end{gather}[#1|end[gather]]
|gdef|@sgatherverbatim#1\end{gather*}[#1|end[gather*]]


|gdef|@multilineverbatim#1\end{multiline}[#1|end[multiline]]
|gdef|@smultilineverbatim#1\end{multiline*}[#1|end[multiline*]]

|gdef|@arraxverbatim#1\end{arrax}[#1|end[arrax]]
|gdef|@sarraxverbatim#1\end{arrax*}[#1|end[arrax*]]

|gdef|@tabulaxverbatim#1\end{tabulax}[#1|end[tabulax]]
|gdef|@stabulaxverbatim#1\end{tabulax*}[#1|end[tabulax*]]


|endgroup
  

  
\def\align{\@verbatim \frenchspacing\@vobeyspaces \@alignverbatim
You are using the "align" environment in a style in which it is not defined.}
\let\endalign=\endtrivlist
 
\@namedef{align*}{\@verbatim\@salignverbatim
You are using the "align*" environment in a style in which it is not defined.}
\expandafter\let\csname endalign*\endcsname =\endtrivlist




\def\alignat{\@verbatim \frenchspacing\@vobeyspaces \@alignatverbatim
You are using the "alignat" environment in a style in which it is not defined.}
\let\endalignat=\endtrivlist
 
\@namedef{alignat*}{\@verbatim\@salignatverbatim
You are using the "alignat*" environment in a style in which it is not defined.}
\expandafter\let\csname endalignat*\endcsname =\endtrivlist




\def\xalignat{\@verbatim \frenchspacing\@vobeyspaces \@xalignatverbatim
You are using the "xalignat" environment in a style in which it is not defined.}
\let\endxalignat=\endtrivlist
 
\@namedef{xalignat*}{\@verbatim\@sxalignatverbatim
You are using the "xalignat*" environment in a style in which it is not defined.}
\expandafter\let\csname endxalignat*\endcsname =\endtrivlist




\def\gather{\@verbatim \frenchspacing\@vobeyspaces \@gatherverbatim
You are using the "gather" environment in a style in which it is not defined.}
\let\endgather=\endtrivlist
 
\@namedef{gather*}{\@verbatim\@sgatherverbatim
You are using the "gather*" environment in a style in which it is not defined.}
\expandafter\let\csname endgather*\endcsname =\endtrivlist


\def\multiline{\@verbatim \frenchspacing\@vobeyspaces \@multilineverbatim
You are using the "multiline" environment in a style in which it is not defined.}
\let\endmultiline=\endtrivlist
 
\@namedef{multiline*}{\@verbatim\@smultilineverbatim
You are using the "multiline*" environment in a style in which it is not defined.}
\expandafter\let\csname endmultiline*\endcsname =\endtrivlist


\def\arrax{\@verbatim \frenchspacing\@vobeyspaces \@arraxverbatim
You are using a type of "array" construct that is only allowed in AmS-LaTeX.}
\let\endarrax=\endtrivlist

\def\tabulax{\@verbatim \frenchspacing\@vobeyspaces \@tabulaxverbatim
You are using a type of "tabular" construct that is only allowed in AmS-LaTeX.}
\let\endtabulax=\endtrivlist

 
\@namedef{arrax*}{\@verbatim\@sarraxverbatim
You are using a type of "array*" construct that is only allowed in AmS-LaTeX.}
\expandafter\let\csname endarrax*\endcsname =\endtrivlist

\@namedef{tabulax*}{\@verbatim\@stabulaxverbatim
You are using a type of "tabular*" construct that is only allowed in AmS-LaTeX.}
\expandafter\let\csname endtabulax*\endcsname =\endtrivlist

% macro to simulate ams tag construct


% This macro is a fix to eqnarray
\def\@@eqncr{\let\@tempa\relax
    \ifcase\@eqcnt \def\@tempa{& & &}\or \def\@tempa{& &}%
      \else \def\@tempa{&}\fi
     \@tempa
     \if@eqnsw
        \iftag@
           \@taggnum
        \else
           \@eqnnum\stepcounter{equation}%
        \fi
     \fi
     \global\tag@false
     \global\@eqnswtrue
     \global\@eqcnt\z@\cr}


% This macro is a fix to the equation environment
 \def\endequation{%
     \ifmmode\ifinner % FLEQN hack
      \iftag@
        \addtocounter{equation}{-1} % undo the increment made in the begin part
        $\hfil
           \displaywidth\linewidth\@taggnum\egroup \endtrivlist
        \global\tag@false
        \global\@ignoretrue   
      \else
        $\hfil
           \displaywidth\linewidth\@eqnnum\egroup \endtrivlist
        \global\tag@false
        \global\@ignoretrue 
      \fi
     \else   
      \iftag@
        \addtocounter{equation}{-1} % undo the increment made in the begin part
        \eqno \hbox{\@taggnum}
        \global\tag@false%
        $$\global\@ignoretrue
      \else
        \eqno \hbox{\@eqnnum}% $$ BRACE MATCHING HACK
        $$\global\@ignoretrue
      \fi
     \fi\fi
 } 

 \newif\iftag@ \tag@false
 
 \def\tag{\@ifnextchar*{\@tagstar}{\@tag}}
 \def\@tag#1{%
     \global\tag@true
     \global\def\@taggnum{(#1)}}
 \def\@tagstar*#1{%
     \global\tag@true
     \global\def\@taggnum{#1}%  
}

% Do not add anything to the end of this file.  
% The last section of the file is loaded only if 
% amstex has not been.



\makeatother
\endinput

\begin{document}

\title{Ionization of molecules by multicharged bared ions by using the \
stoichiometric model }
\author{author}
\affiliation{Instituto de Astronom\'{\i}a y F\'{\i}sica del Espacio (CONICET-UBA).
Casilla de correo 67, sucursal 28 (C1428EGA) Buenos Aires, Argentina.}
\author{author}
\affiliation{Instituto de Astronom\'{\i}a y F\'{\i}sica del Espacio (CONICET-UBA).
Casilla de correo 67, sucursal 28 (C1428EGA) Buenos Aires, Argentina.}
\affiliation{Fac. de Ciencias Exactas y Naturales, Universidad de Buenos Aires}
\date{\today }

\begin{abstract}
\end{abstract}

\pacs{79.20.Rf, 68.49.Bc, 34.20.Cf,34.50.-s,34.35.+a}
\maketitle

\section{INTRODUCTION}

The damage caused by the impact of multicharged heavy projectile has become
a field of interest because of the use in ion-beam cancer therapy. The
effectiveness of the radiation depends on the choice of the ions.\ In
particular theoretical and experimental studies with different projectiles
arrived to the conclusion that charged Carbon ions could be the most
convenient ion to use. This opens a \ new challenge \ for the theoretical
point of view, since multicharged ions at not-so-high-velocities departs
from the first Born approximation which is the most popular theory used in
this field. The first Born approximation as any perturbative direct
calculation warrants the Z$^{2}$ laws where Z is the projectile charge.
However the concentrated damage in the vicinities of the Bragg peak
correspond to energies of hundreds of Kev/amu, that is in the intermediate
energy region. Precisely in this region \ the Born approximation starts to
fail. There is another serious problem from the theoretical point of view
and this is that we are dealing with complex molecules. This article deals
precisely with theses two items. First, we perform more appropriate
calculations on the main damage mechanism -i.e. atomic ionization by
multicharged ions which can replace the Born results. And secondly we use
and inspect the stoichiometric model which reduces a molecule to a sum of
atomic processes quantities weight by the \ numbers of a that atom in the
molecule.

As we have to deal with multicharged ions we resort to the Continuum
Distorted Wave-Eikonal Initial State (CDW for short) which includes higher
perturbative corrections. We start from the premise that ionization process
is the mechanism that deposit the largest primary energy. This process \
produces an electron-energy spectrum that we need to be integrated. In
addition these ejected electrons becomes a new source of local damage which
needs to be investigated because this quantities are used in MonteCarlo
simulations. With this end in mind we not only calculate ionization cross
section but also energy and angular mean values of the emitted electrons

To deal with the molecules we resort to the simplest stoichiometric model
(SSM): this is to consider that the molecule is composed of independent
isolated atoms . In all cases that the total cross section is simply the sum
of cross section of each individual atoms.

With this two tools in mind, i.e. CDW instead of Born approximation and the
SSM, we calculate ionization cross section of different molecules (see Table(%
\ref{20})) by the impact of antiprotons, H$^{+}$, He$^{2+}$, Be$^{4+}$, C$%
^{6+}$,\ and O$^{8+}.$\ In particular \ in section 3, we are interested in
ADN molecules such as Adenine, Timine, Cytosine, Guanine, Uracil, and DNA
backbone.

The results are processed to test the Toburen scaling rule which states that
the ratio of the ionization cross section divided by the number valence
electrons (outer electrons) plotted in terms of the projectile velocity can
be \textbf{grouped} in a narrow universal band. We have also applied this
rule to several hydrocarbons and \ pirimide molecules. We will prove that
the resulting universal band can be greatly sharped (\textbf{angostada}) if
we redefine \ effective valence electrons as developed in Section 3.2

It is well known the biological damage caused by the electron emerging after
the ionization process. To inspect this mechanism, in section 3.3 AND 3.4 we
have calculated the mean electron energy an angular values. To our surprise
we found a substantial dependence of the projectile charged, totally
unexpected in first Born approximation.

To deal with large projectile charge we resort to the CDW but the simplest
stoichiometric model~\ used \ seems to be very simplistic since consider
each atom as neutral, which is not correct. In section 3.5 we run the
chemistry code GAMES to calculate the excess or defect of electron density
one each atoms. With these knowledge we modify the simple stoichiometric
model to account for the departure of the neutrality. We find that, for the
ADN molecules this modification does not introduce a substantial change.

\section{THEORY: IONIZATION OF ATOMS}

For our interest we require just six\ atoms $\alpha =$H, C, N, O, P and S.
Most of the organic molecules are composed by these atoms. \ Some particular
molecules include also halogen atoms such as Fluor and Bromine; ionization
cross sections for these elements in its ion form were already published 
\cite{miraglia2008,miraglia2009}.

Total ionization cross sections of these atoms were calculated using the CDW
The initial bound and final continuum radial wave functions were obtained by
using the RADIALF code developed by Salvat and co-workers using a Hartree
Fock potential obtained the Depurated Inversion Model \cite%
{mendez2016,mendez2018} . The number of pivots used to solve the Schr\"{o}%
dinger equation rounds a few thousands of points, depending on the number of
oscillations of the continuum state. The radial integration was performed
using the cubic spline technique. The number of angular momenta considered
varied between 8, at very low ejected-electron energies, up to 30, at the
largest energies considered. The same number of azimuth angles were required
to obtain the four fold differential cross section \ The calculation so
performed does not display prior-post discrepancies at all. Each atomic
total cross section in Eq. (1) was calculated using 35 to 100 momentum
transfer values, 28 fixed electron angles and around 45 electron energies
depending on the projectile impact energy. Details of the calculation can be
seen in \cite{montanari2017} In parallel we will be reporting state to state
ionization cross sections for these cases in Ref.[\cite{miraglia2019}] which
will be very useful to estimate molecule fragmentation.

In Figure 1 we report our results of total ionization of the six basic
elements by impact of six different projectile, i.e.: antiprotons, H$^{+}$,
He$^{2+}$, Be$^{4+}$, C$^{6+}$,\ and O$^{8+}$ \ To bring all these 36
magnitudes to a consistent figure we exploit the property that in first Born
approximation, the ionization cross section scales with the projectile
charge to square , i.e. Z$^{2}$\ \ The values of impact energies considered
range between 0.1 to 10 MeV/amu, i.e where the CDW is supposed to hold. In
fact for the highest projectile charges the minimum impact energy where the
CDW is expected to be valid could be larger than 100 KeV. From some Mev/amu
on, the first the Born approximation provides quite reliable results and
using a much simpler formalism; we check it. We keep the same color to
indicate the projectile charge along the all figures of this work :
dashed-red, solid-red, blue, magenta, olive and orange for\ antiprotons, H$%
^{+}$, He$^{2+}$, Be$^{4+}$, C$^{6+}$,\ and O$^{8+},$ respectively . It is
notable that there is no complete tabulation of ionization of atoms by
impact of multiple charge ions. We hope that the ones presented in this
article will be very useful.

\section{IONIZATION OF MOLECULES}

\subsection{The stoichiometric model}

Lets us consider a molecule $M$\ composed by $n_{\alpha }$ atoms of the
element $\alpha $, the SSM describes the total ionization cross section of
the molecule $\sigma _{M}$\ as a simple sum of ionization cross sections of
the isolated atoms $\sigma _{\alpha }.$ i.e. 
\begin{equation}
\sigma _{M}=\dsum\limits_{\alpha }n_{\alpha }\sigma _{\alpha }  \label{10}
\end{equation}%
We divided sixteen molecular targets of our interest in 3 families, namely%
\begin{equation}
\begin{array}{|l|l|}
\hline
\text{CH} & \text{CH}_{4}\text{,\ C}_{2}\text{H}_{2}\text{, C}_{2}\text{H}%
_{4}\text{, C}_{2}\text{H}_{6}\text{, and C}_{6}\text{H}_{6}\text{.} \\ 
\hline
\text{CHN} & \text{C}_{5}\text{H}_{5}\text{N, C}_{4}\text{H}_{4}\text{N}_{2}%
\text{,\ C}_{2}\text{H}_{7}\text{N}_{1}\text{, and\ C}_{1}\text{H}_{5}\text{N%
}_{1} \\ \hline
\text{ADN} & 
\begin{array}{l}
\begin{array}{l}
\text{Adenine(C}_{5}\text{H}_{5}\text{N}_{5}\text{),Timine(C}_{5}\text{H}_{6}%
\text{N}_{2}\text{O}_{2}\text{),} \\ 
\text{Cytosine(C}_{4}\text{H}_{5}\text{N}_{3}\text{O),Guanine(C}_{5}\text{H}%
_{5}\text{N}_{5}\text{O),} \\ 
\text{Uracilo(C}_{4}\text{H}_{4}\text{N}_{2}\text{O}_{2}\text{)},%
\end{array}
\\ 
\begin{array}{l}
\text{DNA backbone (C}_{5}\text{H}_{10}\text{O}_{5}\text{P}_{1}\text{)} \\ 
\text{dry DNA(C}_{20}\text{H}_{27}\text{N}_{7}\text{O}_{13}\text{P}_{2}\text{%
)}%
\end{array}%
\end{array}
\\ \hline
\end{array}
\label{20}
\end{equation}

In Figure 2 we report total ionization cross sections by the impact of
multicharged ions for the four basic ADN component that is: \ Adenine,
Timine Cytosine, and Guanine, and in Figure 3 we report results on Uracil
and ADN Backbone containing P. For Adenine the agreement with the three
experiments available in our range of validity are quite good. For Uracil
there is a conflicting situation, we have a good agreement with the
experiments by proton impact \ of itoh et al \cite{itoh2013}, but for the
same target our theory fails by a factor of two for impact of C$^{4+}$ and O$%
^{6+}$ ions . Anyway should be said that our theoretical results coincide
with the ones of Champion Rivarola and collaborators \cite%
{champion2012,agnihotri2012}, indicating a possible\ misstep of the
experiments.

\subsection{Scaling rule}

The first attempt to develop a simple but comprehensive phenomenological
model for electron ejection from large molecules was posed by Toburen and
collaborators \cite{toburen1975,toburen1976} \ These authors found
convenient to scale the experimental ionization cross section in terms of
the number of outer or weakly bound valence electrons (total number of
electrons minus the inner-shell ones). We then introduce the cross section
per weakly bound electron $\sigma _{e}$, or simply\ ionization cross section
per valence electron \ as%
\begin{eqnarray}
\sigma _{e} &=&\frac{\sigma _{M}}{N_{M}}=\frac{\dsum\limits_{\alpha
}n_{\alpha }\sigma _{\alpha }}{\dsum\limits_{\alpha }n_{\alpha }v_{\alpha }}%
=\sigma _{e}(v),\text{ with}  \label{27} \\
v_{\alpha } &=&\left\{ 
\begin{array}{ll}
4,\ \ \  & \text{for C,} \\ 
5, & \text{for N and P} \\ 
6, & \text{for O and S}%
\end{array}%
\right.
\end{eqnarray}%
and \ $v_{H}=1$\ The Toburen rule can be stated by saying that $\sigma _{e}$
is an \textit{universal} parameter independent on the molecule and depending
only on the impact velocity$.$ In Ref. [\cite{toburen1976}] it was found
experimentally that for proton impact on some simple molecules $\sigma _{e}\ 
$result

\begin{equation}
\begin{array}{cccc}
E\text{(MeV)} & \ \ \ 0.25 & \ \ \ 1.00 & \ \ \ 2.00 \\ 
\sigma _{e}\ (\text{in}\ 10^{-16}cm^{2}) & \ \ \ 0.39 & \ \ \ 0.17 & \ \ \
0.11%
\end{array}
\label{30}
\end{equation}%
with an estimated error of about 20\%. A similar ratio was found in Ref.[%
\cite{itoh2013}] dealing with protons on uracil and adenine. Figure 5a shows 
$\sigma _{e}$ calculated with the CDW as a function of the impact velocity
for different projectile charges calculated with the SSM for the sixteen
molecular targets displayed in (\ref{20}). \ The universality with $Z$ is
the one provided by Born \ approximation, i.e. $\sigma _{e}(Z)=Z^{2}\sigma
_{e}(Z=1)$ and it holds for large impact velocities, as shown Figure5a .
Obviously, for lower impact velocities the distorted wave CDW breaks this
behavior discarding the rule\ $Z^{2}$. Although \ the Toburen rule holds for
high energies, its performance is still not satisfactory: the universal band
is quite wide. This departure of our theoretical results from the Toburen
rule can be explained by the simple fact that our atomic cross sections $%
\sigma _{\alpha }$ behaves differently for each atom in terms of the
projectile charges.\ By inspecting Figure 1 one can easily see that the rule 
$\sigma _{\alpha }/v_{\alpha }\sim \sigma _{e}$ is not well satisfied by the
CDW. A much better theoretical atomic rule is given by

\begin{eqnarray}
\sigma _{e}^{CDW} &=&\frac{\dsum\limits_{\alpha }n_{\alpha }\sigma _{\alpha
}^{CDW}}{\dsum\limits_{\alpha }n_{\alpha }v_{\alpha }^{CDW}},~\text{\ with}
\label{32} \\
v_{\alpha }^{CDW} &\sim &\left\{ 
\begin{array}{ll}
4,\ \ \  & \text{for C, N, and O,} \\ 
4.5,\ \  & \text{for P and S}%
\end{array}%
\right. \text{ }  \label{35}
\end{eqnarray}%
and obviously $v_{H}^{CDW}=1.\ \ \sigma _{e}^{CDW}$ is plotted in Figure 5b
where we can observe a much better sharp band, specially at high energy
where the theory is expected to work. Experiments will be interesting to
tune up this theoretical prediction

\subsection{Emitted electron energies}

In a given biological medium, ion direct ionization account for just a
fraction of the overall damage. Secondary electrons as well as recoil target
ions also contribute substantially to the total damage. We can consider the
magnitude $d\sigma _{\alpha nl}/dE$ \ -i.e. the single differential cross
section \ of the shell $nl$ of the atom $\alpha $ as a function of the
ejected electron energy $E$-\ as a simple distribution function \ref%
{surdutovic2018}. We can therefore define\ the mean value $\overline{E}%
_{\alpha }\ $simply as Ref.[\cite{Abril2015}]

\begin{eqnarray}
\overline{E}_{\alpha } &=&\frac{\langle E_{\alpha }\rangle }{\langle
1\rangle }=\frac{1}{\sigma _{\alpha }}\dsum\limits_{nl}\int dE\ E\frac{%
d\sigma _{\alpha ,nl}}{dE}  \label{40} \\
\langle 1\rangle  &=&\sigma _{\alpha }=\dsum\limits_{nl}\int dE\ \frac{%
d\sigma _{\alpha ,nl}}{dE}  \label{50}
\end{eqnarray}%
where $\Sigma _{nl\ }$takes into account the sum of the different sub-shells
contribution of the element $\alpha .$

Figure 5 shows $\overline{E}_{i}$ for our six elements of interest. Our
range of impact velocities in this figure was shorten up to $v=10$ and this
so because of \ a numerical limitation of our spherical harmonics expansion.
In our theoretical treatment we expand our final continuum wave function in
the usual form

\begin{equation}
\psi _{\overrightarrow{k}}^{-}(\overrightarrow{r})=\sum_{l=0}^{l_{\max
}}\sum_{m=-l}^{l}R_{kl}^{-}(r)Y_{l}^{m}(\widehat{r})Y_{l}^{m^{\ast }}(%
\widehat{k}),  \label{60}
\end{equation}%
We are confident with our calculations up to $l_{\max }\sim 30.$ As the
impact velocity $v$\ increases $\langle E_{\alpha }\rangle $\ also
increases, and therefore$\ \ l_{\max }$ should also increases, including in
this way very oscillatory functions in the integrand. Furthermore, as the
integrand of $\langle E_{\alpha }\rangle $ includes the kinetics-energy
variable $E\ ($ see Eq.(\ref{40})$),$ it\ cancels the small energy region
and reinforces the large one making the result more sensible lo large
angular momenta. Anyway for $v>10,$ the first Born approximation holds.

In Figure 5 we can \ estimate $\overline{E}_{\alpha }$ in the range \ 0.5-2
in a.u. or equivalently in between 15-50 eV for all the targets in
accordance with the experimental findings \ref{surdutovic2018} but what it
surprising is the sensible dependence with the projectile charge $Z$\ which
can duplicate the proton results. This effect can be attributed to the
depletion caused by the multicharged ions to the yields of low energy
electrons. In the high \ electron energy regime the CDW falls on the simple
first Born approximation surviving\ the simple $Z^{2}$ law which the ratio
in Eq.(\ref{40}) cancels out, becoming then $\overline{E}_{\alpha }$ an
universal value independent on $Z.$

Extending the simple stoichiometric model for the mean electron energy
calculation, it results%
\begin{equation}
\overline{E}_{M}=\frac{\dsum\limits_{\alpha }n_{\alpha }\overline{E_{\alpha }%
}}{\dsum\limits_{\alpha }n_{\alpha }\sigma _{\alpha }}.  \label{70}
\end{equation}%
For impact of H$^{+}$ and He$^{2+~}$ on water at 1 MeV/amu we obtain of \ $%
\overline{E}_{H_{2}O}=$43.7 \ and 45eV, while the experiments on liquid
water Pimblott and LaVerne [] found \ 51.5 and 52.2 eV, respectively. Just
to stress the importance of the projectile charge, for O$^{8+}$ on water at
the same impact energy 1 MeV/amu we, obtain $\overline{E}_{H_{2}O}=54.7$ eV,
i.e. 25\% larger than proton impact. We recall here that there should not be
any difference if the calculation is carried in first Born approximation. We
will came back on this issue

\subsection{Emitted electron angles}

As mentioned before, secondary electrons contribute to the total damage. It
is important not only the ejection energy but also the angle of emission. In
similar fashion, we can consider the magnitude $d\sigma _{\alpha
,nl}/d\Omega $ \ -i.e. the single differential cross section in terms of the
ejected electron solid angle $\Omega $ -\ as a distribution function, and
define the mean angle $\overline{\theta }_{\alpha }$ as

\begin{equation}
\overline{\theta }_{\alpha }=\frac{\langle \theta _{\alpha }\rangle }{%
\langle 1\rangle }=\frac{1}{\sigma _{\alpha }}\dsum\limits_{nl}\int d\Omega
\ \theta \ \frac{d\sigma _{\alpha ,nl}}{d\Omega }
\end{equation}%
Figure 6 shows $\overline{\theta }_{\alpha }$ for our six elements of
interest. Again here we observe an important dependence dependence of $%
\overline{\theta }_{\alpha }\ $with $Z.$ It is the common view \cite%
{Rudd1992} that the angular dispersion of emitted electrons are nearly
isotropic due to the fact that angular anisotropy of sub-50-eV yield is not
significant. A typical used value of ejection angle is about $\overline{%
\theta }_{\alpha }\sim $70 degrees \cite{surdutovic2018} and this quite
correct in the range of validity of the first Born approximation for any
target. But when a distorted wave approximation is used as observed in
Figure 6, $\overline{\theta }_{\alpha }$ decreases substantially with $Z$ in
the intermediate energy region. For example for C$^{6+}$ impact the Bragg
peak occurs at 0.3 MeV/amu where \ $\overline{\theta }_{\alpha }$ calculated
in CDW is about the half of the one calculated with the First Born
approximation. \ This correction should close the damage to the forward
direction.

The only responsible of this correction is the\ capture to the continuum
effect: the larger the charge $Z$ the smaller $\overline{\theta }$ will be,
of course at intermediate energies, not at high impact energies where the
Born approximation rules. One illustrative observation is the behavior of
antiprotons: the projectile in this case repels the electrons \ making the
distribution almost symmetric. Note the opposite effect of proton and
antiprotons, they run one oppose to the other one, as compared with the
first Born approximation.

\subsection{A modified stoichiometric model}

One can correctly argue that\ the SSM considers an assembly of isolated
neutral atoms which is definitively unrealistic. A first improvement is to
consider that the atoms are not neutral and within the molecule they have
more or less electrons given by a \ charge $q_{\alpha }$. One parameter that
measures $q_{\alpha }$ is the Mulliken charge. This is not the only one,
there is a large variety of charges such as the net or natural atomic charge 
\cite{lee2003}, the lowin charge, etc.

Consider\ that the total amount of electrons $Q_{\alpha }\ $on element $%
\alpha $ are equally distributed on all the atoms $\alpha ,$ therefore each
element $\alpha \ $\ will have a an additional charge: $q_{\alpha
}=Q_{\alpha }/n_{\alpha }$. positive or negative depending on the relative
electronegative value respect to the other ones\ \cite{rappe1991}$.$Instead
of the integer number of elements $n_{\alpha }$ of the atom $\alpha ,$ It
could be argued that we have now an continuous number of atoms\ \ given by 
\begin{equation}
n_{\alpha }^{\prime }=n_{\alpha }-\frac{q_{\alpha }}{v_{\alpha }^{CDW}}
\label{100}
\end{equation}%
In the case of neutral atoms $q_{\alpha }=0,$ we obviously recover $%
n_{\alpha }^{\prime }=n_{\alpha }$ as before$.$

To inspect the effect of the $q_{\alpha }$ we have run GAMES for.... In the
table ( \ref{110}) we display $q_{\alpha }$ of the different ADN basis and
in the last column we rewrite the new stoichiometric expression. 
\begin{equation}
\begin{array}{|l|l|l|l|l|l|}
\hline
\text{Element\ \ } & \text{H} & \text{C} & \text{N} & \text{O} & \text{New\
stoichiometry} \\ \hline
\text{Adenine} &  &  &  &  & \ \ \text{C}_{5}\text{H}_{5}\text{N}_{5} \\ 
\hline
\text{Timine} & +0.14 & +0.07 & -0.22 & -0.38 & \ \text{\ C}_{4.91}\text{H}%
_{5.16}\text{N}_{2.09}\text{O}_{2.13} \\ \hline
\text{Cytosine} &  &  &  &  & \ \text{\ C}_{3.81}\text{H}_{4.46}\text{N}%
_{3.12}\text{O}_{1.11} \\ \hline
\text{Guanine} &  &  &  &  & \ \ \text{C}_{5}\text{H}_{5}\text{N}_{5}\text{O}
\\ \hline
\end{array}
\label{110}
\end{equation}%
All the previous formula holds with the simple replacement of integer value $%
n_{\alpha }$ by the continuous values\ $n_{\alpha }^{\prime }$ to produce a
new ionization cross section $\sigma ^{\prime }$

In Figure 7, we displays the relative difference of ionization cross
sections 
\begin{equation}
e=\frac{\sigma ^{\prime }-\sigma }{\sigma }\times 100  \label{120}
\end{equation}%
for the four ADN basis. The difference is \ just very few percents.
Indicating that the SSM is a quite robust model to treat the type of
molecule we are dealing with, and of course within the range error that one
can expect from the SSM.

We now return to the calculation of the mean electron energy of H$^{+}$ and
He$^{2+~}$ in water\ \ $\overline{E}_{H_{2}O}$. If we consider that \ the
residual charge of \ each hydrogen in water is $q_{H}=$+0.35 \ \cite%
{rappe1991}\ which implies that on Oxygen we have \ a residual charge of $%
q_{O}=$-0.7, we could re-write the formula of water using (\ref{120}) as H$%
_{1.3}$O$_{1.17}$ considering $n_{O}^{CDW}\simeq 4$ and $n_{H}^{CDW}\simeq
1. $ Recalculating mean electron energy by impact of \ H$^{+}$ and He$^{2+~}$
on water at 1 MeV/amu we obtain: $\overline{E}_{H_{2}O}=$46.2 \ and 47.6 eV
which compared a little better with the experiments: 51.5 and 52.2 eV \cite%
{pimblott2007}, respectively. i.e. within 10\% of the experiments which is
not bad accounting for the simplicity of our method.

Another parameter that we can calculate in our model is the fraction of
energy carried by the electron $f$ defined as%
\begin{equation}
f=\frac{\dsum\limits_{\alpha }n_{\alpha }\sigma _{\alpha }\overline{%
E_{\alpha }}}{\dsum\limits_{\alpha }n_{\alpha }(\sigma _{\alpha }\overline{%
E_{\alpha }}+\dsum\limits_{nl}\sigma _{\alpha nl}E_{\alpha nl})}  \label{90}
\end{equation}%
Using our shell-to-shell ionization cross sections $\sigma _{\alpha nl}\ $%
\cite{miraglia2019}\ and atomic binding energies of the atom$\ \alpha ,$ $%
E_{\alpha nl}\ $\cite{clementi}, we obtain $f=$0$.70\ $and $\ 0.71$ for \ H$%
^{+}$ and He$^{2+~}$ on water at 1 MeV/amu,. which differ from the
experiments \ $f=0.81$ and $0.80$, \ respectively.

This difference with the experiment could be attributed to the difference of
binding energy between the molecular water 12.65 eV and the atomic component
H (13.6 eV) and O (17.19 eV), which is an issue beyond our independent atom
approximation and so our model cannot not accounted for. We could estimate
this contribution resorting, once more to the stoichiometric, model, we can
arriesgar a upper-limit difference per molecule of water\ as

\begin{equation}
\delta E=(13.6-12.65)1.17+(17.19-12.65)1.3=7\ \text{eV}  \label{80}
\end{equation}%
Most of this energy should be transfer to the kinetics energy of the emitted
electron. Thus we can \ add 7 eV to the previous values to give $\overline{E}%
_{H_{2}O}=$53.2 \ and 54.6 eV for \ H$^{+}$ and He$^{2+~}$ on water at 1
MeV/amu, respectively, much near to the experiments: 51.5 and 52.2 eV. If we
add $\delta E$ to the numerator of Eq.(\ref{90}) we obtain\ $f=0.81$ and $%
0.80$ in perfect agreement with the experiments (see Table 1 of Ref.(\cite%
{pimblott2007})

\section{CONCLUSIONS}

We have calculated ionization cross sections by impact of antiprotons, H$^{+}
$, He$^{2+}$, Be$^{4+}$, C$^{6+}$,\ and O$^{8+}$ with molecule involved in
biological basis containing H, C, N, O, P and S with the CDW method. \ The
importance of the influence of $Z\ $was observed in the mean energy $%
\overline{E}_{\alpha }\ $and angle $\overline{\theta }_{\alpha }.$ For a
given target $\alpha ,$ as $Z$ increases $\overline{E}_{\alpha }$ increases
but $\overline{\theta }_{\alpha }$ decreases \ At high impact energy, say
larger than 1 MeV/amu these values tend to the ones of the Born
approximation which embodies the simple Z$^{2}$ law. Sixteen molecules were
investigated using the simple stoichiometric model. Results for the six ADN
basis were presented and compared with the sparse available experiments \ We
explore the rule of Toburen which scales all the molecular ionization cross
section when divided by the number of weakly bound valence electrons $\nu
_{\alpha }$ given by Eq.(\ref{27})$.$ We have found the rule scales much
better when normalizing our theoretical ionization cross sections to the
number\ $\upsilon _{\alpha }^{\prime }$ given by Eq.(\ref{35}). And finally
we attempt to improve the stoichiometric model by the use of the Mulliken
charge to redefine a new stoichiometric model containing continuum rather
than integer proportions. NO substantial correction was found indicating
that the SSM works quite well.

Our aims is this article is to provide the tools to estimate any inelastic
parameter parameter -such as the emission angle, the emitted mean energy and
cross section- by the impact of any multicharged \ on any molecule
containing H, C, N, O, P and S, with the help of the stoichimetrical model.
Our goal was quite \textbf{pretencioso}, considering the simplicity of our
proposal. However we think our results could be used to estimate the
ionization magnitude with an acceptable level of uncertailties

\bigskip

\section{Basura}

\begin{thebibliography}{99}
\bibitem{itoh2013} A. Itoh, Y. Iriki, M. Imai, C. Champion, and R. D.
Rivarola,~Cross sections for ionization of uracil by MeV-energy-proton
impact,\ PHYSICAL REVIEW A 88, 052711 (2013)

\bibitem{miraglia2008} J. E. Miraglia and M. S. Gravielle. Ionization of the
He, Ne, Ar, Kr, and Xe isoelectronic series by proton impact. Phys Rev A 
\textbf{78}, 052705 (2008)

\bibitem{miraglia2009} J. E. Miraglia, Ionization of He, Ne, Ar, Kr, and Xe
by proton impact: Single differential distributions. Phys Rev A \textbf{79},
022708 (2009).

\bibitem{} in energy and angles~,

\bibitem{mendez2016} A.M.P. Mendez, D.M. Mitnik, and J.E. Miraglia.
Depurated inversion method for orbital-specific exchange potentials. Int. J.
Quantum Chem. 24 ,116 (2016).

\bibitem{mendez2018} A.M.P. Mendez, D.M. Mitnik, and J.E. Miraglia. Local E
ective HartreeFock Potentials Obtained by the Depurated Inversion Method,
????? 76. (2018).

\bibitem{champion2012} C Champion1, M E Galassi, O Foj\'{o}n, H Lekadir, J
Hanssen1, R D Rivarola,

P F Weck, A N Agnihotri, S Nandi, and L C Tribedi. Ionization of RNA-uracil
by highly charged carbon ions. J. Phys.: Conf. Ser. 373, 012004 (2012).

\bibitem{agnihotri2012} A. N. Agnihotri, S. Kasthurirangan, S. Nandi, A.
Kumar, M. E. Galassi, R. D. Rivarola, O. Foj\'{o}n, C. Champion, J. Hanssen,
H. Lekadir, P. F. Weck, and L. C. Tribedi. Ionization of uracil in
collisions with highly charged carbon and oxygen ions of energy 100 keV to
78 MeV.\ \ Phys Rev A 85, 032711 (2012).

\bibitem{toburen1975} W. E. Wilson and L. H. Toburen. Electron emission from
proton ---hydrocarbon-molecule collisions at 0.3---2.0 MeV. PHYSICAL REVIEW
A\ 11 1303 (1975).

\bibitem{montanari2017} Ionization probabilities of Ne, Ar, Kr, and Xe by
proton impact for different initial states and impact energies. Nucl. Instr.
Meth. Phys. Res. B 407 (2017) 236-243.

\bibitem{miraglia2019} J. E. Miraglia. Shell-to-shell ionization cross
sections of antiprotons, H$^{+}$, \ He$^{2+},$ Be$^{4+},$ C$^{6+}$ and O$%
^{8+}$ on H, C, N, O, P, and S atoms To be published Archive 2019.

\bibitem{toburen1976} D. J. Lynch, L. H. Toburen, and W. E. Wilson. Electron
emission from methane, ammonia, monomethylamine, and dimethylamine by 0.25
to 2.0 MeV protons. J. Chem. Phys. 64, 2616 (1976).

\bibitem{surdutovic2018} Multiscale approach to the physics of radiation
damage with ions. E. Surdutovich and A. V. Solov'yov, arXiv:1312.0897v,
(2013)

\bibitem{abril2015} P. de Vera1, I. Abril, R. Garcia-Molina and
A.V.Solov'yov,\ Ionization of biomolecular targets by ion impact: input data
for radiobiological applications. Journal of Physics: Conference Series 438
(2013) 012015

\bibitem{lee2003} Jung-Goo Lee, Ho Young Jeong, and Hosull Lee, Charges of
Large Molecules Using Reassociation of Fragments. Bull. Korean Chem. Soc.24\
2003, 369 .

\bibitem{rappe1991} A. K. Rappe, A. K.and W. A. Goddard III,. J. Phys. Chem. 
\textbf{95 (}1991) 3358.

\bibitem{pimblott2007} S.M. Pimblott and J. A. LaVerne, ????Radiation
Physics and Chemistry 76, 1244-1247 (2007)

\bibitem{Rudd1992} M. E. Rudd, Y.-K. Kim,, D. H. Madison and T. J. Gay.
Electron production in proton collisions with atoms and molecules: energy
distributions. Rev. Mod. Phys. \textbf{64}, 44-490 (1992).
\end{thebibliography}

\end{document}
